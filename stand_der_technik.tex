\chapter{Stand der Technik}
\label{chap:tech}
Lokalisation wichtiges Element im Bereich der Robotik, besonders Navigation. Andere Forschungsbereiche?
Viele Forschungsgruppen beschäftigen sich mit der Problematik, sowohl Lokalisation in bekannter als auch unbekannter Umgebung.
Da das Ziel des entwickelten Systems in der Lokalisation in bekannter Umgebung liegt wird sich bei der Darstellung der bisherigen Forschungsansätze auf diesen Bereich konzentriert. Da das entwickelte System aufgrund seiner Komponenten dafür ausgelegt ist, die Lokalisation auf Basis von Tiefen- und Farbinformationen durchzuführen werden an dieser Stelle Ansätze, welche auf anderen Verfahren wie Ortung von Funkwellen basieren, ebenfalls nicht näher ausgeführt.\\

Neben globaler Lokalisation auch kontinuierliche Lokalisation (Tracking). Kidnapped Robot Szenario beschreiben.\\

Projektion von Bildinformationen (Augmented Reality Bereich) ebenfalls großes Forschungsgebiet, besonders im Bereich der digitalen Bildverarbeitung (computer vision).\\

Benutzerinteraktion basierend auf der Verwendung von Tiefeninformationen. Hauptsächlicher Ansatz ist die Befehlsvorgabe über Gestensteuerung.\\
In welchen Bereichen wird Augmented Reality angewendet und für welche Anwendungen/in welcher Weise soll es den Benutzer unterstützen?\\


Welche Lokalisationsverfahren gibt es. Allgemein und speziell für handgeführte Systeme.\\
Monte Carlo Lokalisation\\
Featurebasierte Lokalisation (RGB-D SLAM, Fovis)\\
Markerbasierte Lokalisation\\

Welche Systeme gibt es zur Projektion von (Modell-)Daten.\\


Welche Formen von Benutzerinteraktion gibt es, besonders bezogen auf die Kinect und Projektionssysteme.\\
(z.B. Omnitouch)