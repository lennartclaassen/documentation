\chapter{Stand der Technik}
\label{chap:tech}
Die Lokalisation mobiler Systeme in bekannten Umgebungen ist Gegenstand aktueller Forschungsvorhaben.\\
Während die Positionsbestimmung im Freien durch die Verwendung von GPS \\
Durch die ubiquitäre Nutzung von Computern in Form von Smartphones wird die Lokalisation in Gebäuden\\
Besonders in der Robotik hat sich die Selbstlokalisation mobiler Systeme in den letzten Jahrzehnten zu einem wichtigen Forschungsbereich entwickelt, da sie eine Voraussetzung für die autonome Navigation von Robotersystemen darstellt. \red[Andere Forschungsbereiche?] Zu unterscheiden ist dabei zwischen den Verfahren der globalen und lokalen Lokalisation. Bei der globalen Lokalisation soll die absolute Pose\footnote{Die Pose eine Systems umfasst die Beschreibung seiner Position und Orientierung bezogen auf die sechs räumlichen Freiheitsgrade \red[Formel oder Quelle?]} des Systems innerhalb seiner Umgebung ermittelt werden. Die lokale Lokalisation bezieht sich hingegen auf die Bestimmung einer relativen Transformation zwischen dem vorhergehenden und dem aktuellen Zustand. Während die globale Lokalisation damit meist angewendet wird um die initiale Pose z. B. in einer bekannten Karte festzulegen dient die lokale Lokalisation dazu die Veränderungen der Systempose kontinuierlich zu verfolgen.\\
Es werden je nach Anwendungsfall dabei unterschiedliche Ansätze angewendet um eine zuverlässige Lokalisation zu realisieren. Im Folgenden soll ein Überblick über die verbreitetsten Ansätze gegeben werden. Eine weitere Unterscheidung innerhalb dieser Ansätze wird dabei nach der Anzahl der zur Lokalisation betrachteten Posen vorgenommen werden. 
%Unimodale Ansätze berücksichtigen jeweils nur eine Pose, während bei multimodalen Verfahren mehrere Posen gleichzeitig aufrechterhalten werden.
\section{Unimodale Lokalisationsverfahren}
Bei den unimodalen Verfahren wird nur eine Pose des Systems betrachtet, von welcher ausgehend dann die Bestimmung der absoluten oder relativen Lokalisation erfolgt. Da der Einsatz im Rahmen einer globalen Lokalisation damit meist nur bei Vorlage einer \red[(sinnvollen)] Hypothese eines Anfangszustands praktikabel ist, werden unimodale Verfahren vornehmlich für die lokale Lokalisation eingesetzt. \red[Darauf eingehen, welche Verfahren im Rahmen dieser Arbeit interessant sind/warum einige nicht behandelt werden sollen]

\begin{itemize}

\item Scan Matching: Scan matching bezeichnet den Abgleich von Messungen der Umgebungen mit vorangegangenen Messungen \cite{Gutmann1996} oder anderen Vergleichsdaten wie. z.B. einer zuvor aufgezeichneten Karte \cite{Gutmann1998}. Durch Bestimmung der maximalen Überlappung kann die translatorische und rotatorische Veränderung der Pose bezüglich der Referenz berechnet werden. Die betrachteten Messwerte sind dabei meist Distanzen, welche z.B. mit Laser- \cite{Diosi2007} oder Ultraschall-Entfernungsmessern \cite{Burguera2005} aufgezeichnet wurden. Die Berechnung der Transformationsvorschrift zwischen den Zuständen erfolgt auf Basis verschiedener Algorithmen wie dem \textit{Iterative Closest Point} Verfahren (ICP) \cite{Besl1992}\cite{Lu1994} oder der \textit{Normal Distributions Transformation} (NDT) \cite{Biber2003}.

\item Linienmatching: Ein das scan matching erweiternder Ansatz ist das line matching, welches die Tatsache ausnutzt, dass Innenräume häufig geradlinige Strukturen aufweisen. Die Robustheit der Lokalisation kann somit gesteigert werden indem anstelle der einzelnen Messpunkte Linien aufeinander abgeglichen werden. Die Linien werden dabei aus den Messwerten extrahiert und mit zuvor erstellten Liniendaten aus Umgebungskarten verglichen \cite{Cox1991}\cite{Gutmann1999}. Auch ein umgekehrtes Vorgehen ist möglich, in welchem die Liniendaten der Karten in äquidistante Punktmengen transformiert und mit den Daten der Entfernungsmessung abgeglichen werden. In diesem Fall reduziert sich die Berechnung der Überdeckung auf ein abgleichen von Punkten, welches mit den Algorithmen des scan matching gelöst werden kann. Eine Erhöhung der Komplexität wird durch die Betrachtung von Polylinien, also zusammenhängenden Linienelementen erreicht \cite{Wolter2004}. Zur Lösung können dabei Algorithmen angewendet werden, welche für das Abgleichen von Formen entwickelt wurden und ursprünglich aus der computergestützen Bildverarbeitung stammen.\\
%Ähnlich wie das Scanmatching von Punkten, allerdings werden Linien miteinander verglichen, da besonders in Gebäuden etc. die robustheit aufgrund der natürlich vorhandenen Linien gesteigert wird. Das matching kann dabei auf verschiedenen Arten durchgeführt werden: Extraktion von Linien aus den Messdaten (glättet Daten evtl. auch) und Matching auf Kartendaten, Extraktion von (äquidistanten) Punkten aus Kartendaten und Matchinng auf Punktdaten wie beim Scan Matching oder auch Erstellung von Deskriptoren für linien wie länge, winkel, Mittelpunkt.\\
Eine andere Art von Ansätzen des line matchings verfolgen die Beschreibung der Liniendaten mittels verschiedener Deskriptoren wie Länge, Mittelpunkt und Neigungswinkel \red[RICK, auch für QUELLEN!?]. Dieses Vorgehen ermöglicht \red[XXX] und bildet den Übergang zu den merkmalsbasierten Lokalisationsverfahren.

\item Merkmalsbasiert: Die merkmalsbasierte Lokalisation ist eine allgemeinere Anwendung des Prinzips von Deskriptoren (Features) zur Ermittlung der aktuellen Pose eines Systems. Betrachtet werden je nach Anwendungsfall und verwendeter Sensorik unterschiedlichste Merkmale. Neben Distanzmessungen \cite{Tomono2004} werden insbesondere auch Kamerasysteme verwendet um Merkmale aus der Umgebung zu extrahieren \cite{Se2001}. Durch die hier ebenfalls vorhandene Nähe zu Anwendungsfällen in der computergestützten Bildverarbeitung kann auf eine Vielzahl von Algorithmen aus diesem Forschungsbereich zurückgegriffen werden. Da die Definition geeigneter Deskriptoren außerdem bei der Verwendung von Entfernungsmessern häufig nicht generalisierbar ist \red[belegen!?], hat sich für die merkmalsbasierte Lokalisation die Verwendung von Bilddaten als Basis bewährt und wird unter dem Begriff der visuellen Odometrie\red[Fußnote o.a.] zusammengefasst.\\
%Definition der Deskriptoren bei Laserscan etc. schwierig und Anwendungsabhängig -> hauptsächlicher Einsatz von merkmalsbasierter Lokalisation bei visuellen Daten (Kamerabildern) \red[visuelle Odometrie fällt darunter!] SIFT/SURF, PCA.

\item Kalman Lokalisation: Ein weiteres unimodales Lokalisationsverfahren ist die Schätzung und Verfolgung der Systempose mittels eines Erweiterten Kalman-Filters\red[Fußnote oder Beschreibung im Anhang; Monomodale Variante des Bayes Filters]. Dieses Verfahren kann auf den bisher beschriebenen matching Verfahren aufbauen, ist dabei jedoch nicht auf diese limitiert. Durch das Kalman-Filter wird eine Fusionierung der Sensordaten mit der Odometrie erreicht, wobei die Sensordaten z.B. aus globalen Kartenmerkmalen ermittelt werden \cite{Leonard1991} oder aus der Kombination verschiedener Sensoren \cite{Roumeliotis1997} resultieren. Ein großer Vorteil dieses Lokalisationsverfahrens liegt demnach darin, mehrere Datenquellen, sowohl für Odometrie als auch für Messungen integrieren zu können. Da das Kalman-Filter die Pose mittels einer Wahrscheinlichkeitsdichtefunktion annähert eignet er sich besonders unter der Voraussetzung, dass eine Approximation der initialen Pose vorliegt. Die Anwendung in der globalen Lokalisation ohne initiale Startpose ist nur durch parallele Verwendung multipler Kalman-Filter zu realisieren und führt damit zu einem multimodalen Lokalisationsverfahren.
%Anwendung des Kalman-Filters auf mehrere Hypothesen führt zu multimodalen Lokalisationsverfahren. 
%EKF approximiert Verteilung lokal als Gaußverteilung
\end{itemize}
\section{Multimodale Lokalisationsverfahren}
Multimodale Lokalisationsverfahren erhalten stets mehrere mögliche Systemposen aufrecht. Dadurch ermöglichen sie neben einer globalen Lokalisation ohne Anfangshypothese auch das Retten aus falschen Lokalisationen in lokalen Minima. Im Falle einer Fehllokalisation ist z.B. das scan matching zwar in der Lage die Überdeckung zwischen Sensordaten und aktueller Kartenumgebung zu minimieren, der Algorithmus erkennt jedoch nicht, ob sich das System an einer falschen Stelle befindet. Wichtig wird dieser Aspekt insbesondere beim sogenannten \textit{kidnapped robot scenario}, bei welchem das System im Betrieb aus seiner bekannten Pose in eine unbekannten Pose gebracht wird \cite{Yic2011} ohne dabei Sensordaten zu verwerten. Multimodale Verfahren können dem Begegnen, indem sie stets eine Anzahl \textit{n} an wahrscheinlichsten Posen betrachten und/oder in jedem Lokalisationsschritt zufällige Posen in die Betrachtung integrieren.\\
\begin{itemize}
\item Markov Lokalisation: Die Basis der Markov Lokalisation bildet die Diskretisierung des Posenraums. Es wird dabei für jeden Freiheitsgrad eine Rasterkarte erstellt, in welcher die Gitterzellen Repräsentationen möglicher Posen darstellen. Die Wahrscheinlichkeitsdichte wird für jede Zelle auf Basis von Varianten des Bayes Filter bestimmt, zu denen auch das Kalman Filter gehört \cite{Hertzberg2012}. In jedem Lokalisationsschritt werden Odometrie oder Sensordaten verarbeitet und die Wahrscheinlichkeitsdichte der Gitterzellen basierend darauf angepasst. Da selbst bei deutlicher Anhäufung der Wahrscheinlichkeitsdichte um eine Pose auch alle weiteren diskreten Posen weiter betrachtet werden kann es somit nicht passieren, dass das Vertrauen in die Pose mit der aktuell höchsten Wahrscheinlichkeit so groß wird, dass das System nicht in der Lage ist auf falsche Lokalisationen zu reagieren.\\
Die globale Lokalisation erfolgt bei der Markov Lokalisation entweder durch Initialisierung mit einer Gleichverteilung über alle Zellen, oder durch eine Initialisierung mit Normalverteilung und geringer Varianz um die Hypothesen der Startpositionen \cite{Hertzberg2012}.\\

\item Monte Carlo Lokalisation: Die Diskretisierung und gleichzeitige Betrachtung aller möglichen Posen ist mit großem Rechenaufwand verbunden. Die Monte Carlo Lokalisation basiert ebenfalls auf der Diskretisierung des Posenraumes, betrachtet jedoch anstelle aller möglichen Posen nur ausgewählte Stichproben. Jede dieser Stichproben entspricht einer Pose und wird auch als Partikel bezeichnet. Durch die Kontrolle über die Partikelanzahl ist es möglich, den Algorithmus auf die verfügbaren Rechenressourcen abzustimmen. Gleichzeitig kann es natürlich bei zu klein gewählter Partikelanzahl zu einer inkorrekten Lokalisation kommen \cite{Thrun2001}. Aufgrund der relativ einfachen Implementierbarkeit und der geringen Einschränkungen bezüglich der zugrunde liegenden Wahrscheinlichkeitsverteilungen hat sich die Monte Carlo Lokalisation als häufig angewendete Variante etabliert \cite{Thrun2005}.
\end{itemize}

\red[Monte Carlo etc. ->Buch]
\red[smartphone lokalisierung thematisieren aber verwendet externe Sensorik zur Positionsbestimmung]
Viele Forschungsgruppen beschäftigen sich mit der Problematik, sowohl Lokalisation in bekannter als auch unbekannter Umgebung.
Da das Ziel des entwickelten Systems in der Lokalisation in bekannter Umgebung liegt wird sich bei der Darstellung der bisherigen Forschungsansätze auf diesen Bereich konzentriert. Da das entwickelte System aufgrund seiner Komponenten dafür ausgelegt ist, die Lokalisation auf Basis von Tiefen- und Farbinformationen durchzuführen werden an dieser Stelle Ansätze, welche auf anderen Verfahren wie Ortung von Funkwellen basieren, ebenfalls nicht näher ausgeführt.\\

In der Robotik wird die fortlaufende Ableitung der Orientierung und Geschwindigkeit aus Messungen der Raddrehwinkel als Odometrie bezeichnet. \cite{Hertzberg2012}

Neben globaler Lokalisation auch kontinuierliche/lokale Lokalisation (Tracking). Kidnapped Robot Szenario beschreiben.\\

Projektion von Bildinformationen (Augmented Reality Bereich) ebenfalls großes Forschungsgebiet, besonders im Bereich der digitalen Bildverarbeitung (computer vision).\\

Benutzerinteraktion basierend auf der Verwendung von Tiefeninformationen. Hauptsächlicher Ansatz ist die Befehlsvorgabe über Gestensteuerung.\\
In welchen Bereichen wird Augmented Reality angewendet und für welche Anwendungen/in welcher Weise soll es den Benutzer unterstützen?\\


Welche Lokalisationsverfahren gibt es. Allgemein und speziell für handgeführte Systeme.\\
Monte Carlo Lokalisation\\
Featurebasierte Lokalisation (RGB-D SLAM, Fovis)\\
Markerbasierte Lokalisation\\

Welche Systeme gibt es zur Projektion von (Modell-)Daten.\\


Welche Formen von Benutzerinteraktion gibt es, besonders bezogen auf die Kinect und Projektionssysteme.\\
(z.B. Omnitouch)