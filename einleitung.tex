\chapter{Einleitung und Motivation}

%PreviewVersion
%\red[TODO:\\
%Struktur überarbeiten\\
%Ausformulieren\\
%]
%Struktur:\\
%Was ist das Problem, das gelöst werden soll?\\
%Visualisierung in bekannten Umgebungen, AR, Planungsphase von Gebäuden
%Was gibt es für ähnliche Ansätze? Wo kommt der Lösungsansatz her?\\
%Am imes entwickelte Projektor und Lokalisaitonssysteme
%Wie sieht die Lösung aus? Also was wird gemacht?\\
%Daten werden visualisert und modifiziert
%Wie funktioniert die Lösung? Wie wird sie umgesetzt?\\
%kamera-projektoir system zur lokalisatoin und visualisierung und interaktion
%In welcher Weise wird das Problem gelöst?\\
%interaktion merherer beobachter durch projektion möglich, darstellung virtueller Planungsdaten
%Wie ist der Aufbau der Arbeit?\\
%stand der technik, 
%Komponenten, KPS, Software
%lokalisation
%visualisierung
%interaktion
%auswertung
%asublick

Die Visualisierung von Informationen nimmt seit jeher einen großen Stellenwert in der menschlichen Kommunikation und Interaktion ein. Komplexe Sachverhalte lassen sich durch bildliche Darstellung häufig vereinfacht darstellen. Durch Fortschritte in der Computergrafik konnte die Visualisierung virtueller Daten in den Alltag integriert werden. Neue technologische Entwicklungen kombinieren im Rahmen von Virtueller und Erweiterter Realität reale Wahrnehmungen mit computergenerierten Bilddaten.\\

Die Anwendungsbereiche komplexer Visualiserungstechniken reichen von Computerspielen über industrielle Planungsvorgänge bis hin zur Unterstützung bei medizinischen Eingriffen. Mittels neuer Projektions- und Bildschirmkonzepte werden die möglichen Einsatzgebiete dabei stetig erweitert.\\
Am Institut für Mechatronische Systeme der Leibniz Universität Hannover wurde ein handgeführtes Projektionssystem aufgebaut, welches die Visualisierung von Zusatzinformationen in der Chirurgie ermöglicht. Darüber hinaus wurde ein Verfahren zur Registrierung von vermessenen Oberflächen und Modelldaten entwickelt.\\

Aufbauend auf diesen Forschungsarbeiten beschäftigt sich die vorliegende Arbeit mit dem Aufbau eines handführbaren \kps{s}, welches die Darstellung virtueller Zusatzinformationen in der Umgebung ermöglicht.\\
Durch einen Abgleich zwischen Sensor- und Modelldaten der Umgebung wird eine Selbstlokalisation des Systems realisiert. Darauf aufbauend können Zusatzinformationen basierend auf virtuellen Modelldaten in der Umgebung abgebildet und dem Anwender visualisiert werden. Die Interaktion des Benutzers mit den projizierten Daten erlaubt darüber hinaus eine Modifikation der virtuellen Datengrundlage.\\

Das entwickelte Konzept orientiert sich dabei an einem realistischen Anwendungsfall. Im Bauwesen werden Visualisierungen in den virtuellen Planungsprozessen von Neu- und Umbauten genauso wie bei Sanierungen und Innenausbauten von Gebäuden eingesetzt. In allen Bereichen kommt es dabei zur regelmäßigen Kommunikation zwischen den beteiligten Personen, zu denen unter anderem Architekten, Bauingenieure, Techniker und Bauherren zählen. Konzeptions- und Bewertungstreffen innerhalb der Objekte sind deshalb fester Bestandteil der Planungs- und Ausführungsvorgänge.\\
Häufig liegen dabei Innenräume vor, in denen sich wenige weitere Strukturen finden. Das aufgebaute \kps{} erlaubt es, die virtuellen Planungsdaten in diesen Umgebungen zu visualisieren. Die Interaktion der Beobachter mit der Projektion ermöglicht es, Modifikation der Modellumgebung vorzunehmen, welche direkt in den virtuellen Planungsprozess zurückgeführt werden.\\
Die Projektion ist dabei insbesondere als Basis der Kommunikation geeignet, da die Visualisierungen so allen Beteiligten zur gleichen Zeit zugänglich sind.\\

Ein Überblick über den Forschungsstand relevanter Bereiche wird in \kapitel{chap.tech} gegeben. Dabei werden insbesondere Forschungen bezüglich der Lokalisation mobiler Systeme und handgeführter Projektionssysteme betrachtet. In \kapitel{chap.material} werden die technologischen Komponenten und das daraus aufgebaute \kps{} beschrieben. Es wird eine Kalibrierung der Kamera und des Projektors durchgeführt um die Abbildungstransformationen bezüglich der Umgebung zu bestimmen. Anschließend werden die verwendeten Softwarebibliotheken und die entwickelte Programmstruktur erläutert.\\
Die Datengrundlage für die Modellumgebung und die virtuellen Modellobjekte wird in \kapitel{chap.modeldata} erläutert. \kapitel{chap.loc} beschreibt die eingesetzten Lokalisationsverfahren zur Bestimmung und Verfolgung der \red[Pose] des \kps{s} innerhalb der Umgebung. Der Visualisierungsvorgang wird in \kapitel{chap.vis} beschrieben. Dabei wird der Vorgang zur Bestimmung der perspektivisch korrekten Projektion virtueller Objekte und die Ergänzung der Visualiserung um zusätzliche Informationen erläutert.\\
Die Implementierung zur Erfassung der Benutzerinteraktion beschreibt \kapitel{chap.interaction}. Um eine Bewertung der Robustheit und Performanz des Systems zu ermöglichen werden in \kapitel{chap.results} experimentelle Untersuchungen durchgeführt. Dabei wird die Genauigkeit der Lokalisation und Projektion sowie die Latenzzeit der Visualisierung analysiert und bewertet. Abschließend fasst \kapitel{chap.zusammenfassung} die durchgeführten Arbeiten zusammen und gibt eine Ausblick auf mögliche zukünftige Erweiterungen und Optimierungen des aufgebauten \kps{s}.

%Bauwesen als Oberbegriff
%Innenarchitekt(ur), Architekt, Bauingenieur, Bauherr
%Innenausbau, Sanierung, Modernisierung, Umbau
%
%Ziel der Arbeit/Motivation:\\
%Bei der Planung von Gebäuden oder anderen Bauten kann es erforderlich sein die Planungen von bestimmten Strukturen nach Erstellung des Rohbaus innerhalb des realen Modells zu überprüfen. Dies ermöglicht die Erkennung von Planungsfehlern sowie die Änderungen der Planungsdaten und kann somit Verzögerungen während der Bauphase entgegenwirken.\\
%
%Lokalisation in aus Modelldaten bekannter Umgebung.\\
%Um die Planungsdaten überprüfen zu können ist es erforderlich, dass die Position des entwickelten Systems innerhalb der realen Umgebung erkannt wird um einen Abgleich mit den Planungsdaten zu ermöglichen.\\
%Die Modelldaten liegen dabei vor, z.B. als 3D-Modell eines Gebäudes oder Raumes. Ziel ist daher zunächst die Ermittlung der aktuellen Pose des Systems innerhalb der Umgebung. Um die räumliche Lokalisation zu ermöglichen verfügt das System über eine RGB-D Kamera (Microsoft Kinect) welche ein Abbild der Umgebung in Form einer Punktewolke erstellt. Die dadurch ermittelten Tiefendaten können dann dazu verwendet werden einen Abgleich zwischen Modell und Realität durchzuführen. Um die Genauigkeit zu erhöhen und den Rechenaufwand der Lokalisation zu verringern verfügt das System darüber hinaus über eine Inertialmesseinheit (IMU) mit welcher die Orientierung im Raum bezüglich der Drehachsen \glqq Roll\grqq\space und \glqq Pitch\grqq\space bestimmt werden kann.\\
%Sobald die initiale Position des Systems im Raum detektiert wurde, erfolgt ein Tracking der Systemposition durch Fusionierung von visuellen Odometriedaten mit Lage- und Beschleunigungsdaten, welche durch die IMU ermittelt wurden.\\
%
%Projektion von Modellstrukturen.\\
%Nachdem die Position des Systems innerhalb der Umgebung bestimmt wurde, sollen bestimmte Strukturen und Informationen visualisiert werden. Dazu verfügt das entwickelte System über einen Laser-Projektor, mit welchem die gewünschten Visualisierungen auf Wände und Objekte projiziert werden können. Es ist dazu erforderlich die Position bzw. die Perspektive des Projektors innerhalb der Umgebung zu bestimmen. Hierzu wird im Vorfeld eine Kalibrierung des Kamera-Projektor-Systems durchgeführt, sodass die Projektorperspektive durch eine Koordinatentransformation bestimmt werden kann.\\
%
%Detektion von Benutzereingaben.\\
%Die projizierten Informationen sollen dazu dienen, eine Überprüfung und Bewertung der Planungsdaten durch den Benutzer vornehmen zu können. Um eine direkte Korrektur oder Modifikation der Daten zu ermöglichen ist es daher erforderlich, dass der Benutzer mit dem System bzw. der Visualisierung interagieren kann. Das System muss daher in der Lage sein, Benutzereingaben zu erkennen und zu verarbeiten. Dazu wird ebenfalls die Kinect verwendet, um durch Auswertung des Tiefenbildes die Benutzeraktionen zu erkennen.\\
%
%Modifikation des Modells.\\
%Sobald eine Benutzereingabe erkannt wurde, wird diese ausgewertet und entsprechend verarbeitet. Je nach Auswahl können somit die Modellelemente modifiziert werden.\\

%PreviewVersion
%Allgemeine Einführung zu Piko Projektoren und welche Möglichkeiten sie bieten.\\%
%Projektion ermöglicht die Kooperation zwischen Personen, da sie von allen Parteien zu sehen ist!