\chapter{Einleitung und Motivation}

Ziel der Arbeit/Motivation:\\
Bei der Planung von Gebäuden oder anderen Bauten kann es erforderlich sein die Planungen von bestimmten Strukturen nach Erstellung des Rohbaus innerhalb des realen Modells zu überprüfen. Dies ermöglicht die Erkennung von Planungsfehlern sowie die Änderungen der Planungsdaten und kann somit Verzögerungen während der Bauphase entgegenwirken.\\

Lokalisation in aus Modelldaten bekannter Umgebung.\\
Um die Planungsdaten überprüfen zu können ist es erforderlich, dass die Position des entwickelten Systems innerhalb der realen Umgebung erkannt wird um einen Abgleich mit den Planungsdaten zu ermöglichen.\\
Die Modelldaten liegen dabei vor, z.B. als 3D-Modell eines Gebäudes oder Raumes. Ziel ist daher zunächst die Ermittlung der aktuellen Pose des Systems innerhalb der Umgebung. Um die räumliche Lokalisation zu ermöglichen verfügt das System über eine RGB-D Kamera (Microsoft Kinect) welche ein Abbild der Umgebung in Form einer Punktewolke erstellt. Die dadurch ermittelten Tiefendaten können dann dazu verwendet werden einen Abgleich zwischen Modell und Realität durchzuführen. Um die Genauigkeit zu erhöhen und den Rechenaufwand der Lokalisation zu verringern verfügt das System darüber hinaus über eine Inertialmesseinheit (IMU) mit welcher die Orientierung im Raum bezüglich der Drehachsen \glqq Roll\grqq\space und \glqq Pitch\grqq\space bestimmt werden kann.\\
Sobald die initiale Position des Systems im Raum detektiert wurde, erfolgt ein Tracking der Systemposition durch Fusionierung von visuellen Odometriedaten mit Lage- und Beschleunigungsdaten, welche durch die IMU ermittelt wurden.

Projektion von Modellstrukturen.\\
Nachdem die Position des Systems innerhalb der Umgebung bestimmt wurde, sollen bestimmte Strukturen und Informationen visualisiert werden. Dazu verfügt das entwickelte System über einen Laser-Projektor, mit welchem die gewünschten Visualisierungen auf Wände und Objekte projiziert werden können. Es ist dazu erforderlich die Position bzw. die Perspektive des Projektors innerhalb der Umgebung zu bestimmen. Hierzu wird im Vorfeld eine Kalibrierung des Kamera-Projektor-Systems durchgeführt, sodass die Projektorperspektive durch eine Koordinatentransformation bestimmt werden kann.

Detektion von Benutzereingaben.\\
Die projizierten Informationen sollen dazu dienen, eine Überprüfung und Bewertung der Planungsdaten durch den Benutzer vornehmen zu können. Um eine direkte Korrektur oder Modifikation der Daten zu ermöglichen ist es daher erforderlich, dass der Benutzer mit dem System bzw. der Visualisierung interagieren kann. Das System muss daher in der Lage sein, Benutzereingaben zu erkennen und zu verarbeiten. Dazu wird ebenfalls die Kinect verwendet, um durch Auswertung des Tiefenbildes die Benutzeraktionen zu erkennen.

Modifikation des Modells.\\
Sobald eine Benutzereingabe erkannt wurde, wird diese ausgewertet und entsprechend verarbeitet. Je nach Auswahl können somit die Modellelemente modifiziert werden.\\

Allgemeine Einführung zu Piko Projektoren und welche Möglichkeiten sie bieten.\\%
Projektion ermöglicht die Kooperation zwischen Personen, da sie von allen Parteien zu sehen ist!