\chapter{Material und Methoden}
\label{chap:material}
%PreviewVersion
%\red[TODO:\\
%Anwendung nicht als System zur detektion der Projektion, sondern als gekoppeltes System. Dadurch Lichtstärke nur für den Anwender relevant, nicht aber für die Erkennung von Strukturen o.ä.\\
%\kps{} auslagern oder in Material beschreiben?\\
%\kps{} Kalibrierung etc. inkl. Grundlagenteil, wie ausführlich?\\
%GUI Struktur auslagern oder hier beschreiben?\\
%]%
In diesem Kapitel werden die Elemente erläutert, aus welchen das entwickelte \kps{} aufgebaut wurde. Neben den verwendeten technischen Komponenten wird der Aufbau und die Kalibrierung des \kps{s} beschrieben. Darüber hinaus werden die verwendeten Softwarebibliotheken und die damit entwickelte Programmstruktur dargestellt.

\section{Technische Komponenten}
Das erstellte System wurde aus verschiedenen Komponenten aufgebaut, welche im Folgenden näher beschrieben werden sollen. 
%PreviewVersion
%\red[TODO: PC Beschreibung]

\subsection{Microsoft Kinect\textsuperscript{\texttrademark} Sensor}
\label{chap.kinect}
Der von Microsoft (Microsoft Corporation, Redmond, USA) entwickelte Kinect\textsuperscript{\texttrademark} Sensor\footnote{Im Folgenden als Kinect bezeichnet.} ist ein System welches eine natürliche Benutzerschnittstelle für Computer und Spielkonsolen bereitstellt indem es die Steuerung von Programmen und Spielen durch Gesten, Körperbewegungen und Sprachbefehle ermöglicht. Die Kinect verfügt über zwei Kameras, mit denen Videoaufnahmen im Farb- (RGB) sowie im Infrarotbereich (IR) möglich sind.\\
Die Technik zur Ermittlung der Tiefeninformationen stammt von der Firma PrimeSense (PrimeSense LTD, Tel Aviv, Israel) und ist unter anderem durch das Patent \cite{Freedman2008} geschützt. Die IR-Kamera wird dabei in Verbindung mit einem IR-Projektor verwendet um Tiefenbilder zu generieren. Der IR-Projektor projiziert ein irreguläres, bekanntes IR-Muster, welches von der IR-Kamera erkannt wird. Über die Verzerrungen des Musters sind anschließend softwareseitig Rückschlüsse auf die Tiefenwerte der aufgenommenen Szene möglich. Da ein Projektor auch als inverse Kamera betrachtet werden kann \cite{Kimura2007}, entspricht der Aufbau der Kinect dem einer Stereokamera. Da die Bildachsen der IR-Kamera und des IR-Projektors parallel orientiert sind, wird ein Punkt im projizierten Muster auf der in \abb{fig.kinect_depth} dargestellten horizontalen Epipolarlinie $e$ im Kamerabild abgebildet. 

\begin{figure}[ht]
	\begin{center}
		\includesvgnew[1]{images/epipolar_3d_03}
		%\includegraphics[scale=1.0]{spacer}
		\caption{Bestimmung der Tiefenwerte über die Epipolargeometrie}
		\label{fig.kinect_depth}
	\end{center}
	%\vspace*{-8mm}
\end{figure}
%PreviewVersion
%\red[Punktlinien der Linsen mit größerem Abstand! Bildebenen kennzeichnen\\Pfeile auf Linsen\\Punkt als Kreuz!?\\]

Wird der korrespondierende Punkt im Kamerabild identifiziert kann die Disparität $d = x - x'$ berechnet und zusammen mit dem Basisabstand $D$ und der Brennweite $f$ darüber der Tiefenwert des Punktes bestimmt werden:

\begin{equation}
z = \frac{D}{d}\cdot f
\end{equation}

Die bekannte Relation zwischen der RGB- und IR-Kamera ermöglicht abschließend die Zuordnung von Farb- und Tiefenwerten (englisch \textbf{d}epth values). Sensoren dieser Art werden daher auch als RGB-D Kameras bezeichnet.\\

Die Kinect stellt eine kostengünstige Möglichkeit zur parallelen Aufnahme von Tiefen- und Farbinformationen dar. Da seit der Veröffentlichung verschiedene quelloffene Treiber entwickelt wurden eignet sich die Kinect besonders auch für die Anwendung in der Forschung. In dieser Arbeit werden neben den Daten der RGB-Kamera auch die Tiefeninformationen sowie die daraus generierten Punktwolken\footnote{Punktwolken sind Datenstrukturen, welche eine Sammlung multidimensionaler Punkte repräsentieren.
%Als Punktwolke wird die Repräsentation der Tiefenwerte durch eine Menge aus Punkten mit dreidimensionalen Koordinaten bezeichnet.
} verwendet.

%Die Kinect umfasst über die Kamerasysteme hinaus vier Mikrofone zur Sprachsteuerung und einen Motor, welcher die Veränderung des Neigungswinkels ermöglicht. Diese Funktionen sind jedoch nicht Bestandteil des in dieser Arbeit entwickelten Systems.\\

%\red[Da Microsoft selbst keine Angaben über die von ihnen lizensierte Technologie macht stammen die Beschreibungen zur Funktionsweise der Kinect aus den Patentschriften der Firma PrimeSense.]\\

%PreviewVersion
%\red[Datenblatt/Spezifikationen in Anhang!\\]

\subsection{Microvision ShowWX+\textsuperscript{\texttrademark}}%Pico Laser Projektor}
\label{chap.projector}
Der Pico-Laser-Projektor ShowWX+\textsuperscript{\texttrademark} der Firma Microvision (Microvison Inc., Redmond, USA) zeichnet sich neben der geringen Größe dadurch aus, dass als Lichtquellen drei Laser in den Farben rot, grün und blau eingesetzt werden. Die Strahlen werden wie in \abb{fig.projtech} gezeigt durch Optiken kombiniert um alle Farben des sichtbaren Spektrums abzubilden. Der Bildaufbau erfolgt durch Ablenkung des kombinierten Strahls an einem MEMS\footnote{Ein MEMS (Mikro-elektromechanisches System) ist ein technisches System, welches aus Komponenten aufgebaut ist, deren Abmessungen im Mikrometerbereich liegen.}-Spiegel.

\begin{figure}[ht]
	\begin{center}
		\includesvgnew[1]{images/projector_tech_03}
%		\includegraphics[scale=1.0]{projector_tech_02}
		\caption{Projektionsprinzip des Pico-Laser-Projektors}
		\label{fig.projtech}
	\end{center}
	%\vspace*{-8mm}
\end{figure}

Die Verwendung von Lasern führt dazu, dass die Schärfe des Bildes unabhängig vom Abstand zur Projektionsfläche. Der Einsatz von Linsen zur Fokussierung der Strahlen ist daher ebenso wie eine manuelle Anpassung der Fokuseinstellungen nicht erforderlich. Genauere Spezifikationen des ShowWX+\textsuperscript{\texttrademark} finden sich in Anhang \ref{app:projector}.


\subsection{Arduino\textsuperscript{\texttrademark} Nano}
\label{chap.arduino}
Der Arduino\textsuperscript{\texttrademark} Nano\footnote{Im Folgenden als Arduino bezeichnet.} ist ein quelloffenes Mikrocontroller Board, welches durch Schnittstellen in Form von analogen und digitalen Ein- und Ausgängen die Steuerung, Kontrolle und Kommunikation mit elektronischen Komponenten wie Sensoren oder Aktoren ermöglicht. Eine detaillierte Auflistung der Spezifikationen des Arduino findet sich in \anhang{app:arduino}.\\
Die zur Erstellung von Programmen bereitgestellte, ebenfalls quell-offene, Software bildet im Zusammenhang mit der Hardware eine ganzheitliche Entwicklungsumgebung mittels derer sich eine Vielzahl von Projekten unterschiedlicher Komplexität realisieren lassen. Der Arduino verfügt dazu über eine USB-Schnittstelle welche sowohl die Übertragung der erstellten Software auf den Arduino als auch den Empfang von Kommunikationsdaten ermöglicht.\\
Damit zusätzliche Lageinformationen für die Lokalisation des \kps{s} zur Verfügung gestellt werden können wird in dieser Arbeit der Arduino verwendet um eine inertiale Messeinheit in das System zu integrieren, welche im folgenden näher beschrieben wird.

%PreviewVersion
%\red[Auf Berechnung der Lagedaten näher eingehen?]\\
%\red[Buttons eingebaut, kurz ansprechen hier]

%\cite{http://arduino.cc/en/Main/ArduinoBoardNano}

\subsection{Inertiale Messeinheit}
\label{chap.imu}
Die von der Firma InvenSense (InvenSense Inc., San Jose, USA) entwickelte inertiale Messeinheit MPU-9250 ermöglicht die Messung der Beschleunigungsdaten bezüglich der sechs räumlichen Freiheitsgrade des Systems. Aus diesen Daten kann die Lage des Systems bezüglich der Achswinkel bestimmt werden \cite{IMU}.\\
Die Anbindung an das System erfolgt wie zuvor beschrieben durch den Arduino, welcher wiederum die Schnittstelle zu der entwickelten Programmstruktur bildet.\\
%PreviewVersion
%\red[Datenblatt in Anhang!]

%Kompass hier nicht verwendet, aber spätere Verwendung denkbar, bedeutet allerdings, dass Orientierung des Raumes bekannt sein muss (oder irgendwie später hinzugefügt werden kann)
%\cite{http://www.invensense.com/mems/gyro/mpu6500.html}

\subsection{Raspberry Pi\textsuperscript{\texttrademark}}
Der Raspberry Pi\textsuperscript{\texttrademark} Computer\footnote{Im Folgenden als Raspberry bezeichnet.} wurde von der Raspberry Pi Foundation (Caldecote, Vereinigtes Königreich) entwickelt und ist ein ARM\footnote{ARM (Advanced RISC Machines) ist eine von der Firma ARM (ARM Limited, Cambridge, Vereinigtes Königreich) entwickelte Systemarchitektur für Mikroprozessoren, welche sich durch geringen Energiebedarf bei hoher Leistungsfähigkeit auszeichnet.}-basierter Mini-Computer welcher auf einer einzigen Platine aufgebaut ist. Das Ziel der Entwicklung des Raspberry liegt in der Realisierung eines voll funktionsfähigen Computers mit geringen Kosten um die Verbreitung besonders in Schulen und Bildungseinrichtungen zu ermöglichen und so das Erlernen von Programmier- und Computerkenntnissen bei Kindern und Jugendlichen zu fördern. Detaillierte Spezifikationen des Raspberry sind in \anhang{app:raspberry} aufgeführt.\\

Der Raspberry wird für diese Arbeit mit einer angepassten Version des Meta-Betriebssystems ROS (siehe Kapitel \ref{chap:ros}) betrieben um die Anbindung an die entwickelte Programmstruktur zu gewährleisten. Der Raspberry dient dabei dazu, das Signal des zu projizierenden Bildes zu empfangen und über den Projektor darzustellen. Durch die direkte Anbindung des Projektors an den Raspberry entsteht ein gekapseltes System, welches damit in jeder auf ROS basierenden Anwendung eingesetzt werden kann.

\section{Das \kps{}}
Aus den beschriebenen Komponenten wurde das in Bild \ref{fig.kinpro} dargestellte handführbare \kps{} aufgebaut. Die Lage des Projektors wurde dabei so gewählt, dass eine Überdeckung des Projektionsfeldes mit dem Sichtfeld der RGB-Kamera der \kin erreicht wird. Im Zentrum des Systems ist eine Platine positioniert, über welche die Kommunikationsverbindung zwischen der inertialen Messeinheit und dem Arduino realisiert wird.\\

\begin{figure}[ht]
	\begin{center}%
		\includesvgnew[1]{images/kinpro_overview_pathes}%
%		\includegraphics[scale=1.0]{projector_tech_02}
		\caption{Das \kps{}}
		\label{fig.kinpro}
	\end{center}
	%\vspace*{-8mm}
\end{figure}

Die Selbstlokalisation des Systems basiert auf einem Abgleich zwischen Umgebung und Modelldaten. Voraussetzung dafür ist die Kenntnis über die Abbildungstransformation zwischen den Punkten der Umgebung und den Koordinaten der Kinect. Es wird daher eine Kalibrierung der Kameras der Kinect durchgeführt um die Abbildungsgleichungen zu bestimmen.\\

Auch die lagerichtige Projektion in der Umgebung erfordert die Ermittlung von Koordinatentransformationen. Die Pose des Projektors kann aufgrund der fixierten Lage im \kps{} durch eine Transformation zwischen den Projektor-Koordinaten und den Kamera-Koordinaten beschrieben werden. Da der Projektor als inverse Kamera beschrieben ist eine weitere Kalibrierung erforderlich. Neben der Ermittlung der Transformationsvorschrift zwischen Kamera und Projektor kann dabei auch die Abbildungsgleichung zwischen der Bildebene des Projektors und den räumlichen Punkten bestimmt werden.\\
Der Ablauf der sequentiellen Kalibrierung des Gesamtsystems wird im Folgenden näher beschrieben. Zunächst werden dafür einige Begriffe und Notationen definiert, welche im Verlauf dieser Arbeit verwendet werden.\\

% werden darüber hinaus weitere Transformationen bestimmt. Die extrinsische Projektor-Transformation beschreibt die Lage des Projektors bezüglich der Kamera und darüber auch bezüglich der Umgebung. Abbildungen zwischen Umgebungspunkten und Projektorkoordinaten werden über die intrinsische Projektor-Transformation charakterisiert. Bei bekannten Kalibrierungsparametern der Kamera lassen sich beide Transformationen durch eine gekoppelte Kalibrierung des \kps{s} bestimmen. 
%Um diese Transformation zu ermitteln wird eine Kalibrierung der RGB- und der IR-Kamera der Kinect durchgeführt.\\

\subsection{Begriffe und Notationen}
In Anlehnung an die Literatur \cite{Zhang2000} werden 2D-Punkte mit \pteq{x,y} und 3D-Punkte mit \pteq[P]{X,Y,Z} bezeichnet. Auf gleiche Weise werden auch Vektoren und Matrizen beschrieben, ausgenommen der für die Punktbeschreibung verwendeten Notation. Vektoren werden demnach über \vec{v} und Matrizen über \vec{M} beschrieben.\\
Mittels der homogenen Koordinatendarstellung lassen sich Punkte gleichzeitig sowohl translatorisch als auch rotatorisch transformieren. Dazu wird der \textit{inverse Streckungsfaktor} als zusätzliche Komponente eingeführt, für welchen standardmäßig $w=1$ gilt \cite{Nischwitz20111}. Zur Unterscheidung werden homogene Koordinaten damit zu \ptheq{x,y} beziehungsweise \ptheq[P]{X,Y,Z} definiert.\\
Falls nicht mittels eines vorangestellten Index $\ve{i}{p}$ anders gekennzeichnet, bezieht sich die Darstellung von Punkten und Körpern immer auf das globale Koordinatensystem, welches mit $\ks{0}$ bezeichnet wird. Transformationsvorschriften zwischen zwei Koordinatensystemen $\ks{j}$ und $\ks{k}$ werden mit Hilfe der Matrixschreibweise als $\tmat{j}{k}$ ausgedrückt.\\
Die Transformation, die einen Punkt vom globalen Koordinatensystem in das Koordinatensystem des Projektors ($\ks{P}$) abbildet, lässt sich beispielsweise demnach ausdrücken als:

\begin{equation}
\ve{P}{P} = \tmat{P}{0}\ve{0}{P}
\end{equation}

Weiterhin werden die in der Literatur verbreiteten Konventionen für die Darstellung und Orientierung von Koordinatensystemen verwendet: Die farbliche Zuordnung der Koordinatenachsen richtet sich nach der RGB-Darstellung. Die $x$-Achsen werden somit rot, die $y$-Achsen grün und die $z$-Achsen blau eingefärbt abgebildet. Bei Kameras und Projektoren wird die Orientierung der körpereigenen Koordinatensysteme darüber hinaus so gewählt, dass die $z$-Achse in Richtung der optischen Achse ausgerichtet wird. Die gewählten Konventionen sind in \abb{fig.coords} anhand des \kps{s} dargestellt.

\begin{figure}[ht]
	\begin{center}%
		\includesvgnew[1]{images/coordinate_systems_KS}%
		%\includegraphics[scale=1.0]{coordinate_systems}
		\caption{Darstellung von Koordinatensystemen, Punkten und Transformationen am Beispiel des \kps{s}}
		\label{fig.coords}
	\end{center}
	%\vspace*{-8mm}
\end{figure}

%PreviewVersion
%\red[Punkte in KS einfügen!\\Punkte als Kreuze?\\]
%\red[Koordinatensysteme nennen!?\\]

\subsection{Kamerakalibrierung}
%Ziel der Kalibrierung ist es, die Transformation $\tmat{S}{0}$ zu bestimmen, welche die Punkte der realen Umgebung $\ve{0}{P}$ auf die Sensorkoordinaten der Kamera $\ve{S}{p}$ abbildet:
Ziel der Kalibrierung ist es, die Funktion $f$ zu bestimmen, welche die Punkte der realen Umgebung $\ve{0}{P}$ auf die Sensorkoordinaten der Kamera $\ve{S}{p}$ abbildet:

\begin{equation}
f : \; \ve{0}{P} \mapsto \ve{S}{p}
\end{equation}

%PreviewVersion
%\red[Wie verschiedene Kameras unterscheiden, KRGB und KIR, Sensorebene: SRGB und SIR und für Projektor SP?\\]

Unter Verwendung des Lochkameramodells \red[\cite{Jaehne2012}], welches von einer Kamera ohne Optik mit infinitesimal kleiner Blendenöffnung ausgeht, kann diese Abbildung beschrieben werden über:

\begin{equation}
s \cdot \ve{S}{\tilde{p}} = \tmat{S}{0}\ve{0}{\tilde{P}}
\label{eq.persp_abb}
\end{equation}

Dabei wird über den Skalierungsfaktor $s$ die Unbestimmtheit der Tiefeninformationen beschrieben, da alle Punkte auf der Verbindungslinie zwischen $\ve{0}{\tilde{P}}$ und $\ve{S}{\tilde{p}}$ ebenfalls auf $\ve{S}{\tilde{p}}$ abgebildet werden.\\ 
%PreviewVersion
%\red[Bild, dass die Uneindeutigkeit zeigt?\\]%
Die Transformationsmatrix $\tmat{S}{0}$ beschreibt das Produkt aus der intrinsischen $\tmat{S}{K}$ und extrinsischen $\tmat{K}{0}$ Matrix der Kamera:

\begin{equation}
\tmat{S}{0} = \tmat{S}{K} \cdot \tmat{K}{0}
\end{equation}

Durch die Kamerakalibrierung können die Parameter dieser beiden Matrizen bestimmt werden. Die intrinsischen Parameter der Kamera sind konstant, während die extrinsische Kameramatrix in der später Anwendung durch die Lokalisation ermittelt wird. Im Rahmen der Kamerakalibrierung dient sie daher lediglich zur Bestimmung der intrinsischen Parameter.\\
Die folgende Beschreibung des Kalibriervorgangs gilt sowohl für die RGB- als auch für die IR-Kamera der Kinect. Die Abläufe unterscheiden sich lediglich bezüglich der aufgenommenen Bilder, da die Kalibrierung der IR-Kamera die Ausleuchtung der Umgebung durch eine IR-Kamera erfordert.

\subsubsection{Extrinsische Kameramatrix}
\label{chap.perspproj}
%PreviewVersion
%\red[Bilder der Kalibrierung?\\]
Die extrinsische Kameramatrix definiert die Transformation zwischen dem globalen Koordinatensystem und dem Koordinatensystem der Kamera. Sie gibt damit die Lage von $\ks{K}$ im $\ks{0}$ an und wird über eine Translation und eine Rotation beschrieben.\\
Während die Translation allgemein über den Verschiebungsvektor zwischen den Koordinatensystemen $\vec{t} = [t_x, t_y, t_z]^T$ angegeben wird, existieren für die Definition der Rotation verschiedene Konventionen.\\

Allgemein kann jede Rotation im dreidimensionalen Raum durch eine Drehung um eine definierte Achse beschrieben und durch eine $(3 \times 3)$ Matrix ausgedrückt werden. Diese Matrix ist eindeutig, Unterschiede in den Konventionen ergeben sich daher lediglich aufgrund der gewählten Repräsentation.\\
Bei der Verwendung körpereigener Systeme ist es sinnvoll die Rotation als Verknüpfung von Elementardrehungen darzustellen, da so eine direkter Zusammenhang zu den Koordinatenachsen des Körpers erzielt wird. Dabei werden Winkel für ausgewählte Achsen definiert und die elementaren Rotationsmatrizen durch Multiplikation verknüpft. Diese Winkel werden auch als Eulersche Winkel bezeichnet \cite{Foley1990}. Zu beachten ist, dass selbst innerhalb dieser Darstellung verschiedene Konventionen existieren. Die Unterschiede beziehen sich dabei auf die Beschreibung der Rotationen anhand mitgedrehter oder fixierter Achsen sowie die Reihenfolge der verknüpften Elementardrehungen.\\

In dieser Arbeit erfolgt die Beschreibung unter Verwendung der ($z$,$y'$,$x''$)-Konvention, welche unter anderem in der Fahrzeugtechnik gebräuchlich ist. Jede Drehung wird dabei anhand der Achsen des körperfesten und damit veränderlichen Koordinatensystems beschrieben. Die erste Rotation wird um den Winkel $\Phi$ um die $z$-Achse durchgeführt (Gieren), gefolgt von einer Drehung um den Winkel $\Theta$ um die $y$-Achse (Nicken). Abschließend wird um den Winkel $\Psi$ um die $x$-Achse rotiert (Rollen). Durch die Verknüpfung der Elementardrehungen lässt sich damit die Rotationsmatrix ausdrücken:

\begin{equation}
\begin{split}
\vec{R} & = \matz{\Phi} \maty{\Theta} \matx{\Psi} \\[1em]
& = \vec{R}_z(\Phi) \vec{R}_y(\Theta) \vec{R}_x(\Psi)
\end{split}
\end{equation}

Durch die Vereinigung von Translation und Rotation in der homogenen Transformationsmatrix $\tmat{K}{0}$ ergibt sich die extrinsische Kameramatrix:

\begin{equation}
\tmat{K}{0} = 
\mat{ccc|c}{
  & \rmat{K}{0} &   & \ve{K}{t} \\
\hline
0 &      0      & 0 & 1 \\
}
\end{equation}

Ein in globalen Koordinaten bekannter Punkt lässt sich damit im Koordinatensystem der Kamera abbilden:

\begin{equation}
\ve{K}{\tilde{P}} = \tmat{K}{0}\ve{0}{\tilde{P}}
\end{equation}

%\red[Rotationsmatrix und Translationsvektor vorher beschreiben und Punktabbildung definieren, dann hom. Trafo]

\subsubsection{Intrinsische Kameramatrix}
Ist die Lage eines Punktes im Kamera-Koordinatensystem bekannt, so lässt sich mittels der intrinsischen Kameramatrix die Abbildung der dreidimensionalen Koordinaten auf die Pixelkoordinaten $\ve{S}{p} = [u,v]^T$ der Sensorfläche der Kamera beschreiben.\\
Mit Rückblick auf das Modell der Lochkamera treffen die Strahlen an einem Punkt $\ve{K}{{p}}$ auf die Bildebene der Kamera in welcher der Sensor sitzt. Die Lage des Punktes ist abhängig von der Brennweite $f$, welche den Abstand zwischen Blendenöffnung und Bildebene angibt. Die Abbildung auf eine Ebene führt zu einer Reduktion der Dimension, wodurch die Bildkoordinaten nur noch zweidimensional angegeben werden können. Dies führt zu dem in \eq{eq.persp_abb} eingeführten Skalierungsfaktor $s$, durch welchen die Unbestimmtheit der Tiefeninformationen gekennzeichnet ist.\\

Die Transformation der metrischen Koordinaten in Pixelkoordinaten erfolgt abschließend durch Verschiebung des Koordinatensystems in den Sensorursprung $[u_0,v_0]$ und Berücksichtigung der Umrechnungsparameter $s_x$ und $s_y$, welche sich aus der Geometrie der Sensorelemente ergeben.\\
Die Abbildung kann mit der intrinsischen Kameramatrix $\tmat{S}{K}$ anhand dieser Parameter beschrieben werden als

\begin{equation}
s \cdot 
\mat{c}{
u\\
v\\
1
}
= 
\mat{cccc}{
s_x \cdot f & 0 & u_0 & 0\\
0 & s_y \cdot f & v_0 & 0\\
0 & 0 & 1 & 0
}
\mat{c}{
X\\
Y\\
Z\\
1
}
\label{eq.intrinsic}
\end{equation}
\begin{equation}
s \cdot \ve{S}{\tilde{p}} = \tmat{S}{K}\ve{K}{\tilde{P}}
\end{equation}

Im Weiteren werden die Produkte aus Brennweite und Umrechnungsfaktoren in den gemeinsamen Parametern $f_x := s_x \cdot f$ und $f_y := s_y \cdot f$ zusammengefasst, da diese im Rahmen der folgenden Kalibrierung nicht unabhängig voneinander zu bestimmen sind.

\subsubsection{Verzeichnungen}
In der Realität ist es erforderlich Objektive mit Linsen einzusetzen um das Lochkameramodell anzunähern, gleichzeitig aber ausreichende Belichtung zu gewährleisten. Es kommt dadurch jedoch zu Verzeichungen, resultierend aus der Form der Linsen und ihrer Lage zueinander. Die Verzeichnung mit dem häufig stärksten Einfluss ist die radiale Verzeichnung, welche durch die Krümmung der Linsen entsteht. Parallele Strahlen laufen dabei nicht in einem Brennpunkt zusammen, wodurch es in Abhängigkeit des Radius zu den in \abb{fig.distortions} dargestellten Verzeichnungen kommen kann \cite{Hertzberg2012}.

\begin{figure}[!ht]
	\begin{center}
	\subfigure[Kissenförmige Verzeichnung]{
		\includesvgnew[0.25]{images/verzeichnungen_kissen}
		\label{fig.verzKiss}
	}
	\hspace{4mm}
	\subfigure[Abbildung ohne Verzeichnung]{
		\includesvgnew[0.25]{images/verzeichnungen_normal}
		\label{fig.verzNorm}
	}
	\hspace{4mm}
	\subfigure[Tonnenförmige Verzeichnung]{
		\includesvgnew[0.25]{images/verzeichnungen_tonnen}
		\label{fig.verzTonn}
	}
	\caption{Auftretende Formen der Verzeichnung und ihre Auswirkungen auf die Abbildung}
	\label{fig.distortions}
	\end{center}
	%\vspace*{-8mm}
\end{figure}

%\includesvgnew[0.25]{images/verzeichnungen_kissen}

Darüber hinaus treten weiter Formen von Verzeichnungen auf, die aufgrund ihres geringen Einflusses jedoch häufig vernachlässigt werden können und in dieser Arbeit nicht näher spezifiziert werden sollen. Die Verzeichnungsparameter können zusammen mit den intrinsischen Parametern im Rahmen einer Kalibrierung ermittelt werden woraufhin sich die Verzeichnungen softwareseitig durch geeignete Modelle korrigieren lassen. 
%so dass eine Abbildungsgenauigkeit im Subpixelbereich erzielt werden kann \red[\cite{}]. Die Verzeichnungsparameter können zusammen mit den intrinsischen Parametern im Rahmen einer Kalibrierung ermittelt werden. \red[Tangentiale Verzeichnungen werden aber auch berechnet in Kalibrierung!]

\subsubsection{Kalibriervorgang}
Basierend auf dem von Zhang \cite{Zhang2000} vorgestellten Verfahren lassen sich die intrinsischen Kameraparameter bestimmen und die radialen Verzeichnungen modellieren. Die Kalibrierung erfolgt dabei durch Betrachtung eines planaren Musters aus verschiedenen Kameraperspektiven. Im folgenden werden die Schritte erläutert, welche zur Bestimmung der Parameter erforderlich sind.\\

Zunächst wird ein geeignetes Muster erstellt und auf eine planare Fläche aufgebracht. In dieser Arbeit wird die Kalibrierebene wie in \abb{fig.chesscalib} gezeigt durch ein auf einen ebenen Untergrund aufgeklebtes Schachbrettmuster definiert.

\begin{figure}[ht]
	\begin{center}
		\includegraphics[scale=0.5]{chessboard_small}
		\caption{Durch Schachbrettmuster definierte Kalibrierebene}
		\label{fig.chesscalib}
	\end{center}
	%\vspace*{-8mm}
\end{figure}

Aus verschiedenen Lagen werden anschließend mit der Kamera Bilder des Musters aufgenommen. Dabei spielt es keine Rolle, ob die Lage der Kamera oder des Musters zwischen den Aufnahmen variiert wird. Zu beachten ist jedoch, dass das Muster in allen Aufnahmen vollständig im Bild abgebildet wird. Darüber hinaus sollten sowohl translatorische als auch rotatorische Veränderungen der Pose vorgenommen werden um signifikant verschiedene Perspektiven abzubilden.\\

Die Kalibrierung erfordert die Bestimmung der Ebene des Musters und darauf befindlicher, eindeutig zu identifizierender Punkte. Das Schachbrettmuster eignet sich besonders als Kalibriermuster, da zwischen den Feldern ein hoher Kontrast besteht. Die Eckpunkte können somit robust erkannt und extrahiert werden, woraufhin sich die Kalibrierebene bestimmen lässt.\\
%PreviewVersion
%\red[\abb{fig.chessplane} Vielleicht chesscalib (b)!? Oder 3 verschiedene Posen mit Koordinatensystem!?]\\

Die 4 intrinsischen und die 6 extrinsischen Parameter werden zunächst auf Basis des Lochkameramodells ohne Berücksichtigung der radialen Verzeichnungen bestimmt. Die Abbildung eines Punktes auf die Sensorebene kann unter Verwendung der intrinsischen und extrinsischen Kameramatrix nach \eq{eq.persp_abb} ausgedrückt werden als:

\begin{equation}
s \cdot 
\mat{c}{
u\\
v\\
1
}
=
\mat{cccc}{
f_x & 0 & u_0 & 0\\
0 & f_y & v_0 & 0\\
0 & 0 & 1 & 0
}
\mat{ccc|c}{
  & \rmat{K}{0} &   & \ve{K}{t} \\
\hline
0 &      0      & 0 & 1 \\
}
\mat{c}{
X\\
Y\\
Z\\
1
}
\end{equation}

Eine Vereinfachung dieser Darstellung lässt sich erreichen, indem die Kalibrierebene mit der globalen Ebene in $Z=0$ gleichgesetzt wird. Dadurch können die Dimensionen der intrinsischen und extrinsischen Kameramatrix reduziert werden. Die Abbildungsgleichung ergibt sich damit unter Umformulierung der Rotations-Matrix als $\rmat{K}{0} = [\vec{r}_1 \quad \vec{r}_2 \quad \vec{r}_3]$ zu:

\begin{equation}
s \cdot
\mat{c}{
u\\
v\\
1
}
 = 
\underbrace{
\mat{ccc}{ 
	f_x & 0 & u_0\\
	0 & f_y & v_0\\
	0 & 0 & 1
}
\mat{cc|c}{ 
	\vec{r}_1 & \vec{r}_2 & \ve{K}{t} \\
}
}_H
\mat{c}{
	X\\
	Y\\
	1
}
\end{equation}

Die $(3 \times 3)$ Matrix $\vec{H}$ wird als Homographie-Matrix bezeichnet und beschreibt die Abbildung zwischen den Punkten zweier Ebenen. Aufgrund des frei wählbaren Skalierungsfaktors $s$ besitzt die Homographie-Matrix bei neun Einträgen insgesamt acht Freiheitsgrade. Da jeder detektierte Punkt des Musters zwei Korrespondenzen liefert, ergeben sich aus vier Punkten damit acht Gleichungen, aus welchen ein Gleichungssystem zur Lösung der Homographie-Matrix erstellt werden kann.\\
%\red[ Homographie beschreibt Transformation eines Quadrilaterals, damit würde auch die Bestimmung von mehr als vier Punkten keine zusätzlichen Informationen liefern.]\\

Da die Betrachtung des Musters aus verschiedenen Perspektiven erfolgt, müssen zunächst für jede Aufnahme die sechs räumlichen Freiheitsgrade bestimmt werden. Es verbleiben damit jeweils zwei Parameter zur Bestimmung der intrinsischen Parameter. Da die intrinsische Kameramatrix nach \eq{eq.intrinsic} vier Freiheitsgrade besitzt, werden $n = 2$ Aufnahmen benötigt um die intrinsischen Parameter eindeutig festlegen zu können. In der Praxis sollten jedoch $n \geq 10$ Aufnahmen verwendet werden um numerische Stabilität der Lösung zu gewährleisten und Messrauschen zu kompensieren \cite{Bradsky2008}.\\

%Homographie bildet 2D auf 2D ab; Gesamtabbildung angeben, reduzieren durch z = 0; Homographiematrix aufstellen; Anzahl freier Parameter nennen (8) -> mindestens 2 Aufnahmen erforderlich; Besser mehr als 10; \red[Formeln!?]; Verweis auf Lösung reicht eigentlich, Parameter können damit bestimmt werden; Dann weiter zur Bestimmung der (radialen) Verzeichnungsparameter.
%Obwohl theoretisch $n = 2$ Aufnahmen für eine eindeutige Lösung ausreichend sind, sollten für eine robuste Bestimmung der Parameter insgesamt $n \geq 10$ Bilder aufgezeichnet werden \cite{Bradsky2008}.\\

Basierend auf den ermittelten intrinsischen Parametern können nun die radialen Verzeichnungsparameter angenähert werden. Unter der Annahme, dass die radiale Verzeichnung klein ist, kann davon ausgegangen werden, dass die intrinsischen Parameter bereits hinreichend genau bestimmt wurden. Die ermittelten Werte können damit als initale Approximation verwendet werden um die Verzeichnungsparameter zu bestimmen. In einem abschließenden Optimierungsschritt werden alle ermittelten Parameter verfeinert. Dazu erfolgt eine Minimierung der Fehler zwischen den aufgenommenen und den auf Basis des ermittelten Modells bestimmten Pixelwerten des Musters.\\

%Es erfolgt eine Verfeinerung der Parameter durch Minimierung der \red[GLEICHUNG!?].%


%\red[Vereinfachung später, dass T-WELT = T-0]
%Kalibrierung der Kamera
%Kameramatrix aufführen, Herleitung ist aber nicht wirklich relevant. Extrinsische und intrinsische Transformation unterscheiden. Parameter beschreiben und erklären. Lochkameramodell als Grundlage nennen und in ein paar Sätzen erläutern, aber nicht zu detailliert drauf eingehen. Ermittlung der Tiefeninformationen beschreiben. projektor als inverse Kamera.\\

%PreviewVersion
%\red[Kalibrierung der Kameras (RGB + IR) mit cameracalibration beschreiben und Gesamtsystem mit Matlab box bouguet (gibt extr. und intr. parameter).\\
%Isometrische Ansicht als Prinzip der Kalibrierung? Schachbrett fest und 3 Ansichten des Systems (der Kamera) mit Koordinatensystemen!?\\]

\subsection{Projektorkalibrierung}
\label{chap.projcalib}
Um die Darstellung visueller Informationen an definierten Stellen im Raum zu ermöglichen ist eine Kalibrierung des Projektors erforderlich. Indem die Projektion in umgekehrter Richtung betrachtet wird, kann ein Projektor wie in \kapitel{chap.kinect} erwähnt als inverse Kamera aufgefasst werden. Das Verfahren zur Kamera-Kalibrierung lässt sich dadurch auf den Projektor erweitern \cite{Falcao2008}. Aufgrund der in \kapitel{chap.projector} beschriebenen Projektionstechnologie können die Verzeichnungen dabei vernachlässigt werden, da keine Optik zur Fokussierung des Bildes verwendet wird. Die Kalibrierung kann damit auf die Bestimmung der intrinsischen und extrinsischen Parameter des Projektors beschränkt werden.\\

Aufgrund der Fixierung des Projektors im \kps{} ist die Lage bezüglich des Kamera-Koordinatensystems unveränderlich. Anstelle einer globalen Positionsbestimmung ist daher insbesondere die Ermittlung der konstanten, relativen Transformation zwischen den Koordinatensystemen der Kamera und des Projektors von Interesse. Ziel der Kalibrierung des Projektors ist daher die Bestimmung der konstanten Parameter der intrinsischen und extrinsischen Projektormatrix.\\

Die Kalibrierung erfolgt anhand des von Falcao \textit{et al.} beschriebenen Verfahrens \cite{Falcao2008}. dabei wird zur Definition der Kalibrierebene erneut ein Muster auf einer Platte verwendet. Ein weiteres Kalibriermuster wird als digitales Bild erstellt und wie in \abb{fig.projcalib} gezeigt durch den Projektor neben das reale Muster auf die Platte projiziert.

\begin{figure}[ht]
	\begin{center}
		\includegraphics[scale=0.5]{chessboard_projected_small}
		\caption{Reales und projiziertes Schachbrettmuster zur Kalibrierung des Projektors}
		\label{fig.projcalib}
	\end{center}
	%\vspace*{-8mm}
\end{figure}

%PreviewVersion
%\red[Bild 90 grad rotieren\\]

Um die gemeinsame Kalibrierebene zu bestimmen ist die Aufnahme beider Muster in einem Bild erforderlich. Da durch die Kalibrierung auch die Transformation zwischen den Koordinatensystemen der Kamera und des Projektors bestimmt werden soll, werden die Muster mit der zuvor kalibrierten Kamera des Systems aufgezeichnet.\\

Zunächst ist es erforderlich, die Lage der Kamera bezüglich der Kalibrierebene zu ermitteln. Mit Hilfe der bestimmten intrinsischen Parameter können die globalen Koordinaten der Eckpunkte und die aufgespannte Ebene des realen Schachbretts leicht bestimmt werden. Die Pixel-Koordinaten des projizierten Musters können ebenfalls extrahiert werden. Da die Kalibrierebene bekannt ist, kann der in \eq{eq.persp_abb} eingeführte Skalierungsfaktor in den Abbildungsgleichungen bestimmt und somit die räumliche Position der projizierten Eckpunkte ermittelt werden.\\

Es liegen nun wie bei der Kamerakalibrierung Korrespondenzen zwischen der Kalibrierebene und der gedachten Sensorebene des Projektors vor. Damit ist das bereits beschrieben Kalibrierverfahren nach Zhang anwendbar, wodurch die intrinsische Projektormatrix $\tmat{SP}{P}$ bestimmt werden kann.\\

%PreviewVersion
%\red[Auch hier isometrische Ansicht mit Gesamtsystem? Transformationen lassen sich dort gut zeigen\\]

Da für jede Aufnahme während der Kalibrierungsvorgänge die Transformationen zwischen dem globalen Koordinatensystem und den Koordinatensystemen von Kamera und Projektor ermittelt wurden, kann die statische Transformation zwischen Kamera und Projektor ebenfalls bestimmt werden:

\begin{equation}
\tmat{K}{P} = \left( \tmat{0}{K} \right)^{-1} \tmat{0}{P} = \tmat{K}{0} \tmat{0}{P}
\end{equation}

Da das \kps{} damit als Gesamtsystem kalibriert ist wird ein gemeinsames Basis-Koordinatensystem $\ks{KPS}$ definiert. Dieses unveränderliche Koordinatensystem dient als Referenz um die externen Transformationen zu beschreiben. Die Orientierung wird entsprechend der ($z$,$y'$,$x''$)-Konvention gewählt und die Position wie in \abb{fig.coords_kps} dargestellt in Abhängigkeit der Kinect definiert. Die Transformation $\tmat{KPS}{K}$ ergibt sich damit aus den geometrischen Beziehungen, woraufhin sich $\tmat{KPS}{P}$ ebenfalls bestimmen lässt.


\begin{figure}[ht]
	\begin{center}%
		\includesvgnew[1]{images/coordinates_kps}%
		%\includegraphics[scale=1.0]{coordinate_systems}
		\caption{Basis-Koordinatensystem des \kps{s}}
		\label{fig.coords_kps}
	\end{center}
	%\vspace*{-8mm}
\end{figure}

%PreviewVersion
%\red[
%%Notation\\
%%Lochkameramodell\\
%%Intrinsische + Extrinsische Parameter\\
%%Kalibrierung Kamera (RGB+IR)\\
%%Stereosystem Kinect (Epipolargeometrie etc.), Prinzip kurz erläutern, Buch und Davids Arbeit - Weiter vorne schon?\\
%%Projektor als inverse Kamera\\
%%Kalibrierung Projektor + Kamera+Projektor\\
%%Projektor Linsenverzeichnung!?\\
%Toolboxen und verwendete Kalibrierungsnode nennen?\\
%Transformationen zwischen den Koordinatensystemen angeben wenn möglich (oder in Kapiteln erst?)]\\

%\red[Gemeinsames Koordinatensystem des Gesamtsystems definieren, über welches die Lage in der Umgebung beschrieben wird. $\ks{KPS}$\\]

%\red[Wichtig sind die Transformationen zwischen den Koordinatensystemen und die Abbildung des Projektors. Die Transformation erfolgt allerdings immer über die Kamera, da diese im Weltkoordinatensystem lokalisiert wird. Für die Kalibrierung sind die intrinsischen und extrinsischen Parameter relevant, aber auch die Herkunft?\\]

%PreviewVersion
%\red[Transformationen:\\
%Map = World\\
%World -> BaseLink -> CameraLink -> Camera\\
%Camera -> Projector\\
%Projector -> IntrinsicProj\\
%World -> Objects\\
%World -> ARMarker (eq. to Objects)\\
%]

\section{Verwendete Softwarebibliotheken/Software}
%\red[GUI auslagern als eigenen Punkt!?\\]
Zur Realisierung der einzelnen Systemfunktionen wurden Komponenten erstellt, welche auf verschiedenen Softwarebibliotheken aufbauen.
Bevor die Programmstruktur näher erläutert wird sollen deshalb zunächst die verwendeten Bibliotheken und Softwareumgebungen dargestellt werden.
Anschließend erfolgt eine Beschreibung der erstellten Programmstruktur und der innerhalb einer grafischen Benutzeroberfläche zusammengefassten Bedienfunktionen des Systems.

\subsection{ROS}
\label{chap:ros}
Das Robot Operating System (ROS) ist eine speziell für die Anwendung in der Robotik entwickelte, quelloffene Sammlung aus Softwarebibliotheken \cite{ROS}. Neben Treibern für verschiedene Hardwarekomponenten und speziellen Algorithmen bietet ROS eine Umgebung, welche die Integration von und Kommunikation zwischen verschiedenen Modulen vereinfacht um komplexe und robuste Anwendungen zu realisieren. 

%PreviewVersion
%\red[Hertzberg Buch detaillierter]

\subsection{Open Source Computer Vision Library}
Die Open Source Computer Vision Library (OpenCV) ist eine quelloffene Bibliothek aus Funktionen und Algorithmen zur Anwendung in der Bild- und Videoverarbeitung \cite{OpenCV}. Nachdem OpenCV ursprünglich von einer Forschungsgruppe bei Intel entwickelt wurde \cite{Laganiere2011}, wird die Bibliothek heute von einer großen Anzahl an Firmen und Entwicklern verwendet und ständig weiterentwickelt. Sie umfasst mittlerweile mehr als 2500 optimierte Algorithmen zur Anwendung in Bereichen wie Objekterkennung, Segmentierung oder Navigation.

\subsection{Point Cloud Library}
Die Struktur von Punktwolken wird verwendet um die von der Kinect aufgenommenen Tiefeninformationen darzustellen und weiterverarbeiten zu können. Die Point Cloud Library (PCL) wurde mit dem Ziel entwickelt, ein Rahmenwerk zu schaffen, welches die Verarbeitung von Punktwolken mittels verschiedener Algorithmen ermöglicht. Ähnlich wie OpenCV für die 2D Bildverarbeitung wird PCL bezogen auf Punktwolken in den Bereichen Objekterkennung, Segmentierung oder Filterung angewendet. Auch PCL ist eine quelloffene Bibliothek, welche von einer Vielzahl an Firmen und Entwicklern ständig überarbeitet und erweitert wird \cite{PCL}.

\subsection{Qt}
Die von der Firma Trolltech entwickelte und mittlerweile von der Firma Digia verwaltete Qt Bibliothek ermöglicht eine plattformunabhängige Entwicklung von grafischen Benutzerschnittstellen im C++ Standard \cite{Qt}. Die Qt Bibliothek wurde in dieser Arbeit verwendet um die Schnittstelle zu realisieren mit welcher der Benutzer Zugriff auf die Funktionen der verschiedenen Programmkomponenten erhält.

\subsection{Visualization Toolkit}
\label{chap.vtk}
Das Visualization Toolkit (VTK) stellt eine auf dem C++ Standard basierende Bibliothek dar, welche für die Verarbeitung und Visualisierung von 3D Bilddaten entwickelt wurde. Die Firma Kitware entwickelt die Bibliothek ständig weiter und stellt sie als quelloffene Software zur Verfügung \cite{VTK}. Verschiedene Schnittstellen zwischen VTK und PCL ermöglichen neben der Darstellung von 3D-Objekten auch die Integration und Visualisierung von Punktwolken. 

%PreviewVersion
%\red[Ausführlicher?\\]
%\red[Versionen!\\]

\subsection{Programmstruktur}
\label{chap.softwarestruct}
%\red[Mit aufnehmen in dieser Liste? Oder ganz am Ende (nach Interaktion) Komplettes GUI und alle Transformationen zusammenfassen?\\]%
Die für die Anwendung des \kps{s} als selbstlokalisierendes, handgeführtes Projektionssystem entwickelte Software umfasst verschiedene Module, welche durch ROS verknüpft werden. Dies ermöglicht den Informationsaustausch zwischen den Komponenten auf Basis definierter Daten- und Kommunikationsstrukturen. Der Austausch einzelner Komponenten ist damit jederzeit möglich, sodass die Entwicklung und Integration optimierter Module je nach Anwendungsgebiet vorgenommen werden kann.\\
Einen %\red[abstrahierten] 
Überblick über die Softwaremodule, ihre Verknüpfung untereinander und die Beziehungen zur realen und modellierten Umgebung zeigt Abbildung \ref{fig.modules}.\\

\begin{figure}[ht]
	\begin{center}%
		\includesvgnew[1]{images/modules_solo}%
		\caption{Übersicht der Softwaremodule}
		\label{fig.modules}
	\end{center}
	%\vspace*{-8mm}
\end{figure}

%\begin{figure}[ht]
%	\begin{center}
%		\includegraphics[scale=1.0]{spacer}
%		\caption{Funktionsmodule der Anwendungssoftware}
%		\label{fig.modules}
%	\end{center}
%	%\vspace*{-8mm}
%\end{figure}

Das Modul \textit{Lokalisation} umfasst die Implementierungen zur Bestimmung der Pose des \kps{s} innerhalb der realen Umgebung. Es empfängt die Farb- und Tiefeninformationen der Kameras der Kinect und ermittelt durch den Abgleich mit der Modellumgebung die aktuelle Systempose. Definiert wird die Pose dabei durch den Benutzer über Bewegung des \kps{s}.\\
Die kontinuierlich bestimmte Pose wird an das Modul \textit{Visualisierung} übertragen, welches daraufhin die Lage des \kps{s} innerhalb der Modellumgebung ermittelt. Durch Bestimmung der perspektivisches Abbildung ausgewählter Modelldaten wird anschließend die Projektorsicht simuliert und dem Benutzer über das \kps{} durch Projektion visualisiert.\\ 

%Dazu werden im Modul \textit{Visualisierung} alle Transformationen bestimmt, welche die Pose  der Modellumgebung definieren.\\
Das Modul \textit{Interaktion} ermöglicht dem Benutzer die Modifikation der visualisierten Modelldaten indem die Tiefeninformationen der Kinect ausgewertet werden. Die Befehle des Benutzers können so erkannt und anschließend an das Modul \textit{Visualisierung} übertragen werden. Durch die Rückkopplung zu den Modelldaten können diese daraufhin angepasst werden.\\
Das Modul \textit{Visualisierung} integriert die Benutzerinteraktion in die Darstellung der Modelldaten, so dass eine visuelle Rückmeldung während der Interaktion erfolgt.\\

Darüber hinaus wurde eine grafische Benutzeroberfläche (GUI) implementiert, welche verschiedene Konfigurationsmöglichkeiten bietet. Sie ermöglicht damit dem Benutzer direkten Zugriff auf die Einstellungen der Module. Diese übergeordnete Kommunikationsstruktur wurden in der Übersicht nicht explizit aufgeführt, da die Anwendung unabhängig von der Benutzeroberfläche lauffähig ist.\\

\red[GUI nomenclature]

%PreviewVersion
%\red[In den folgenden Kapiteln erfolgt eine Spezifizierung der Module und ihrer Komponenten.\\]
%\red[Wie KPS, Benutzer und Modelldaten bezeichnen?\\]

%Module: \mLocalization - EKF Odometrie (Tracking) - Visuelle Odometrie - IMU - Visualisierung - Transformation - Interaktion - Benutzeroberfläche - Projektion - Kartenserver\\

%Das \mLocalization bestimmt und überwacht die aktuelle Position des \kps{s}. Es ermittelt die initiale Position mittels globaler Lokalisation auf Basis der vom \mMapserver zur Verfügung gestellten Karte. Als weitere Eingangsgröße verwendet es die Odometriedaten, welche vom \mEkf zur Verfügung gestellt werden um Veränderungen in der Position während der Bedienung zu erkennen und die Positionsdaten dementsprechend zu aktualisieren. Das \mEkf selbst verwendet und filtert dabei die visuellen Odometriedaten des \mFovis und die Lagedaten des \mImu \red[TODO Moduls] um die kombinierten Odometriedaten zu bestimmen.\\
%Diese so vom \mLocalization bestimmten Positionsdaten werden dem \mTransformation übermittelt, welches alle Koordinatentransformationen zwischen der ermittelten Kameraposition, dem Projektor und den relevanten Objekten der Umgebung berechnet. Diese Berechnungen werden dann vom \mVisualization verwendet, um ein Abbild der aktuellen Projektorsicht innerhalb der 3D Umgebung zu erstellen. Dem Benutzer wird diese Ansicht anschließend sowohl über die Benutzeroberfläche durch das \mGui als auch über die Projektion durch das \mProjection dargestellt. \\
%Das \mInteraction erkennt vom Benutzer durchgeführte Gesten oder Bewegungen, über welche er auf Basis der projizierten Daten mit der Modellumgebung interagiert. Die Steuerungsbefehle werden dann über das \mTransformation an das \mVisualization übertragen, wo diese interpretiert werden und in Form von Modifikationen der Modellumgebung umgesetzt werden.\\
%Die jeweilige Realisierung der Module wird in den folgenden Kapiteln näher spezifiziert.\\


%PreviewVersion
%\red[Buttons eingebaut, hier als Möglichkeit der Interaktion erwähnen(?)- Aber welches Modul managed das?]

