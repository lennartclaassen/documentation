\chapter{Visualisierung}
\label{chap.vis}
\red[TODO:\\
Ergänzung von Zusatzinformationen noch etwas ausfürhlicher?\\
Karte einblenden implementiert?\\
]

Sobald die aktuelle Position des \kps{s} innerhalb der Umgebung ermittelt wurde kann die Visualisierung der gewünschten Zusatzinformationen basierend auf den Modelldaten (siehe \kapitel{chap.modeldata}) erfolgen. Die ausgewählten Modelldaten sollen mittels des integrierten Projektors perspektivisch korrekt in die Umgebung projiziert werden. Dazu ist zunächst die Visualisierung der Modellumgebung und die Ermittlung der Pose des Projektors innerhalb dieser erforderlich. Anschließend erfolgt die Simulation der Projektorsicht durch eine perspektivische Transformation. Die Simulation der Perspektive erfolgt dabei ausgehend von einem Kameramodell, da der Projektor wie in \abschnitt{chap.proj_transformation} als inverse Kamera betrachtet werden kann.\\
Abschließend werden basierend auf der simulierten Perspektive Bilddaten generiert, welche mittels des Projektors für den Anwender und Beobachter innerhalb der realen Umgebung visualisiert werden.

\section{Visualisierung der Modelldaten}
Die Modellumgebung und die Modellobjekte bilden die Grundlage der visuellen Zusatzinformationen die dem Anwender bereitgestellt werden sollen. Die Visualisierung gliedert sich dabei in zwei Bereiche: die Visualisierung des Modells innerhalb der grafischen Benutzeroberfläche und die Projektion der Modellobjekte innerhalb der realen Umgebung. Um die Projektion zu ermöglichen ist es zunächst erforderlich die gesamten Szene durch die Modelldaten abzubilden. Aufbauend auf der in \kapitel{chap.modeldata} beschriebenen Datenstruktur kann eine Szene erstellt oder geladen und anschließend modifiziert werden. Um eine gerenderte\red[footnote?] Abbildung der dreidimensionalen Modell-Szene zu erhalten wird die \red[Visualisierungsbibliothek] VTK verwendet. \abb{fig.modscene} zeigt die Darstellung der gerenderten Modell-Szene innerhalb des Visualisierungsbereiches der grafischen Benutzeroberfläche.

\begin{figure}[!ht]
	\begin{center}
		\includegraphics[scale=1.0]{spacer}
		\caption{Modellszene im GUI}
		\label{fig.modscene}
	\end{center}
	%\vspace*{-8mm}
\end{figure}


Die grafische Benutzeroberfläche ermöglicht dabei die individuelle Positionierung und Ausrichtung aller Elemente. Zu beachten ist, dass die Pose der Modellumgebung selbst nicht modifiziert werden sollte, da wie in \kapitel{chap.map} beschrieben die Lokalisation auf den Daten der Modellumgebung basiert. Der Zusammenhang zwischen den Koordinaten der Modellumgebung und der Karte der Lokalisation wäre bei einer Veränderung der Pose nicht mehr gegeben und würde zu einem Fehler in der späteren Visualisierung führen. Die grafische Benutzeroberfläche enthält die Möglichkeit eine Zuweisung der Modellelemente zu den Gruppen Umgebung und Objekt vorzunehmen, wodurch die Möglichkeiten zur Modifikation der Elemente entsprechend eingeschränkt oder freigegeben werden.


\red[Bilder für Modellumgebung und Modellobjekte (z.B. Steckdosen, Leitungen)]

\section{Ermittlung der Projektorposition}
Durch die in \abschnitt{chap.proj_calibration} durchgeführte Kalibrierung des \kps{s} ist die externe Transformation zwischen der Kamera des Kinect Sensors und dem Projektor bekannt. Die kontinuierlich durchgeführte Lokalisation des \kps{s} liefert die Pose der Kamera innerhalb der Umgebung worüber sich somit auch die Pose des Projektors bestimmen lässt. Ist diese bekannt, kann durch die Kopplung zwischen den Kartendaten der Lokalisation und den verwendeten Modelldaten die Pose des Projektors innerhalb der simulierten Umgebung bestimmt werden.

\section{Perspektivische Transformation}
Die perspektivische Abbildung einer 3D Umgebung auf eine 2D Ebene wurde bereits in \kapitel{chap.perspproj} beschrieben. Der Projektor wird eingesetzt um Daten zu visualisieren, welche zwar innerhalb der Modellumgebung, nicht jedoch in der realen Umgebung vorhanden sind. Der Projektor kann deshalb in der Modellumgebung als Kamera simuliert werden, wodurch die Abbildung der Perspektive und Visualisierung dieser möglich wird. Die perspektivische Transformation der Modelldaten erfolgt damit basierend auf den aus durch die Lokalisation bestimmten extrinsischen und den bei der Kalibrierung ermittelten intrinsischen Parametern des Projektors. Die Abbildung der 3D Modellpunkte auf die 2D pixel der simulierten Kamera wird demnach beschrieben durch:

\begin{equation}
Transformation. der. Umgebungspunkte. in. Projektorkoordinaten.
\end{equation} 

\abb{fig.perspproj_vtk} zeigt die Abbildung eines Modellobjektes auf die Bildebene des als Kamera simulierten Projektors. Die Bildebene ist dabei wie bereits in \abschnitt{chap.pinhole} beschrieben am Brennpunkt gespiegelt dargestellt. Bei Generierung der Abbildung ist zu beachten, dass lediglich die Modellelemente sichtbar sind, die mit dem später projizierten Bild visualisiert werden sollen\red[ (im Bild grün hervorgehoben)]. Die Modelldaten aller bereits vorhandenen Strukturen können deshalb ausgeblendet werden, um ungewünschten Überlagerungen zu vermeiden. Um die Projektions- oder Lokalisationsgenauigkeit zu überprüfen können allerdings bewusst Überlagerungen von Modelldaten und realen Strukturen erzeugt werden.

\begin{figure}[!ht]
	\begin{center}
		\includegraphics[scale=1.0]{spacer}
		\caption{Perspektivische Abbildung der Modelldaten auf die Sensorebene/Pixeldaten des als Kamera simulierten Projektors, grün hervorgehoben das Objekt, das nicht in der Umgebung vorhanden ist aber projiziert werden soll; Rest ausgrauen oder so?}
		\label{fig.perspproj_vtk}
	\end{center}
	%\vspace*{-8mm}
\end{figure}

\red[Darstellung der Perspektive innerhalb des VTK GUIs]

\section{Projektion}
Nachdem durch die perspektivische Transformation ein Bild der Projektorsicht vorliegt erfolgt die Visualisierung der Modellstrukturen in der realen Umgebung. Da bereits alle nötigen Transformationen des Bildes im Rahmen der perspektivischen Abbildung durchgeführt wurden, kann das vorliegende Bild direkt als Ausgangssignal des Projektors verwendet werden. Darüber hinaus kann das Bild um weitere Informationen ergänzt werden. \abb{fig.proj_rms} zeigt, wie das Ausgabebild mit einem Rahmen versehen wurde, welcher dem Anwender eine visuelle Rückmeldung über die Güte der Lokalisation basierend auf dem berechneten RMS Wert (siehe \kapitel{chap.globloc}) gibt.
Projektion der Sicht des Projektors auf das Modell. Ausblenden des Raumes, sodass nur die gewünschten Strukturen dargestellt werden. Alternativ Raummodell als schwarzes Objekt darstellen.

\begin{figure}[!ht]
	\begin{center}
		\includegraphics[scale=1.0]{spacer}
		\caption{Visuelle Rückmeldung für den Anwender über die Lokalisationsgüte}
		\label{fig.proj_rms}
	\end{center}
	%\vspace*{-8mm}
\end{figure}