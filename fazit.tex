\chapter{Zusammenfassung und Ausblick}
\label{chap:zusammenfassung}
\red[TODO:\\
Entwickeltes System kurz zusammenfassen, beginnend damit warum es erstellt wurde!\\
Ergebnisse und Fazit daraus kurz zusammenfassen\\
Ausblick geben auf erweiterungen des Systems - Um die Einsschränkungen zu beheben - Um das System zu erweitern/verbessern\\
Ausblick geben auf mögliche Anwendungsfälle/-gebiete\\
]

\kps{} erstellt zur Darstellung visueller Zusatzinformationen;\\
Selbstlokalisation durch Partikelfilter verwendet;\\
Tracking basierend auf visueller Odometrie verwendet;\\
Robustheit durch IMU und Kalman-Filter gesteigert;\\
Projektion von Modelldaten, Projektor als inverse Kamera, die Modellszene betrachtet, eigentlich nicht mehr direkt invers dann;\\
Gesamtsystem kalibriert;\\
Grafische Benutzeroberfläche erstellt, welche die Funktionen zugänglich macht, ist aber für Anwendung nicht unbedingt erforderlich, dann muss vorher die Welt definiert werden;\\
Buttons integriert;\\
Anbindung an ROS ermöglicht Austausch verschiedener Module;

System ermöglicht Lokalisation und Darstellung visueller Zusatzinformationen;\\
Anwendung unterliegt ein paar Einschränkungen, die jedoch im geplanten Anwendungsszenario keine Probleme machen sollten; Unter Berücksichtigung der Empfehlungen aus dem vorhergehenden Kapitel;\\

Erweiterungen:\\
Kinect V2;
Kompass zur Lagebestimmung yaw -> deutliche Reduzierung der Partikelanzahl möglich;
Gestenerkennung -> Kinect V2 gut, da bessere Auflösung;
