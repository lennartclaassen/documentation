\chapter{Datengrundlage}
\label{chap.modeldata}
\prever{
%\red[TODO:\\
%%Verschiedenen Dateitypen nennen?\\
%%Verwendete Programme nennen? Binvox, blender etc.?\\
%%Dateistruktur -> Sichern von Daten/Rückkopplung des veränderten Modells hier einfügen?\\
%]
}

Das Ziel der Anwendung liegt in der lagerichtigen Darstellung von Modelldaten durch das \kps{} innerhalb der realen Umgebung. Als Grundlage ist es erforderlich, ein möglichst originalgetreues Abbild der Umgebung zu erstellen. Basierend auf dieser Modellumgebung kann die Lokalisation des Systems und die Einblendung virtueller Objekte in der realen Umgebung realisiert werden. In diesem Kapitel wird die Erstellung einer geeigneten Modellumgebung und der zu visualisierenden Modellobjekte erläutert. Darüber hinaus wird eine Datenstruktur definiert, welche die Repräsentation erstellter Modellszenarien ermöglicht.

%Die zu visualisierenden Modellobjekte werden anhand der virtuellen Umgebungsdaten orientiert und positioniert und können über einen Abgleich in die reale Umgebung übertragen werden. 
%Die Projektion der Modellumgebung selbst innerhalb der realen Umgebung würde lediglich zu Überlagerungen gleicher Strukturen führen und besitzt außerhalb von Validierungen des \kps{s} keine Relevanz.\\

\section{Modellumgebung}
\label{chap.slam}
Für die Verwendung der erstellten Programmstruktur ist es erforderlich, dass die Umgebung in Form eines 3D-Modells abgebildet wird. Als Modellumgebung werden dabei alle statischen Objekte und Strukturen betrachtet, welche eine Repräsentation der realen Umgebung darstellen. Dabei können entweder vorhandene Modelldaten als Basis verwendet oder ein Abbild der Umgebung aus Sensordaten erzeugt werden.\\

Liegen bereits Informationen bezüglich der Umgebung vor, beispielsweise in Form von CAD-Daten, kann das virtuelle Umgebungsmodell direkt aus diesen abgeleitet werden. Dies ist insbesondere bei der Anwendung des \kps{s} im Rahmen von Planungs- und Bauphasen von Gebäuden der Fall. Eine beispielhafte Modellumgebung ist in \abb{fig.mapCADSLAM} (a) dargestellt.\\

\nomenclature[a]{CAD}{Computer Aided Design}

\begin{figure}[ht]
	\begin{center}
		\subfigure[CAD-Modell der Umgebung]{
			\includegraphics[scale=0.9]{033_cad_02}
		}
		\hspace{5mm}
		\subfigure[Abbild der realen Umgebung in Form einer Punktwolke]{
			\includegraphics[scale=0.25]{pointcloud_meshlab}
		}
		\caption{CAD-Modell und aus Sensordaten erzeugtes Abbild der Umgebung}
		\label{fig.mapCADSLAM}
	\end{center}
	%\vspace*{-8mm}
\end{figure}

Existieren keine Modelldaten der Umgebung, oder sind diese nicht mehr aktuell, kann ein virtuelles Abbild auf Basis von Sensordaten erzeugt werden. Diese sogenannte Kartierung von 2D- und 3D-Umgebungen ist in aktuellen Forschungen eng mit der Lokalisation innerhalb dieser Umgebungen verknüpft. Ein verbreitetes Forschungsfeld in diesem Zusammenhang ist das simultane Kartieren und Lokalisieren (SLAM\footnote{\textit{Simultaneous Localization And Mapping} (SLAM) bezeichnet die Erstellung eines Abbildes der Umgebung bei gleichzeitiger Bestimmung der Systempose innerhalb dieser.}) in unbekannten Umgebungen. Dabei existieren verschiedene Algorithmen und Implementierungen zur Lösung der Problemstellungen. Die Kartierung selbst soll in dieser Arbeit nicht näher erläutert werden, für eine detailliertere Ausführung sei deshalb auf \cite{Durrant2006} verwiesen.\\

\nomenclature[a]{SLAM}{Simultaneous Localization and Mapping}

Um die experimentelle Analyse des Systems zu ermöglichen wurde eine reale Versuchsumgebung kartiert. Dabei wurde eine Implementierung \cite{Rgbdslam} nach \textsc{Endres} \textit{et al.} \cite{Endres2014} verwendet, welche die SLAM Anwendung mittels einer RGB-D Kamera realisiert. Das \kps{} kann somit auch zur Kartierung der Umgebung eingesetzt werden. Das Ergebnis ist in Form der aufgezeichneten, texturierten Punktwolke in \abb{fig.mapCADSLAM} (b) dargestellt.\\

%\begin{figure}[ht]
%	\begin{center}
%		\includegraphics[scale=0.25]{pointcloud_meshlab}
%		\caption{Abbild der realen Umgebung in Form einer Punktwolke}
%		\label{fig.mapSLAM}
%	\end{center}
%	%\vspace*{-8mm}
%\end{figure}

%\red[Boden und Decke nicht modelliert, Limitierungen der Kartenerstellung -> nicht wirklich relevant und führt zu ungenauer Abbildung\\]

Im Gegensatz zu der Ableitung aus vorhandenen Modelldaten handelt es sich bei kartierten Umgebungen nicht um geschlossene Geometrien. Ungenauigkeiten in der Abbildung können damit durch die im Vergleich zu Modelldaten geringere Auflösung und die Rauscheinflüsse des Entfernungssensors auftreten. Die Güte eines Modells aus der Kartierung liegt damit meist unter der Modellgüte einer aus vorhandenen 3D-Daten abgeleiteten Darstellung. Um die Fehlereinflüsse zu verringern ist es ratsam, dieselben Sensoren während der Kartierung und der Lokalisation zu verwenden, da wiederkehrende Sensoreinflüsse so bei beiden Verfahren auftreten.\\

Die erstellte Modellumgebung dient als Basis der späteren Lokalisation und kann zur Visualisierung der Umgebung innerhalb der grafischen Benutzungsschnittstelle verwendet werden. Die Projektion der Modellumgebung hingegen besitzt außerhalb von Validierungen des \kps{s} keine Relevanz, da sie innerhalb der realen Umgebung lediglich zu Überlagerungen gleicher Strukturen führen würde.

\section{Modellobjekte}
Die Projektion visueller Zusatzinformationen basiert auf virtuellen Modelldaten. Als Modellobjekte werden daher alle Strukturen und Objekte bezeichnet, welche keine Elemente der realen Umgebung abbilden. Diese können wie die Modellumgebung aus vorhandenen Modelldaten abgeleitet oder durch Verarbeitung von Sensordaten modelliert werden. Anstelle von Algorithmen zur Kartierung der Umgebung existieren jedoch spezielle Verfahren zur Erfassung dreidimensionaler Objekte, wie beispielsweise von \textsc{Xu} \textit{et al.} \cite{Xu2012} beschrieben. Mögliche Modellobjekte umfassen im Hinblick auf den betrachteten Anwendungsfall beispielsweise Trockenbauelemente oder Elektroinstallationen wie Leitungen und Steckdosen.\\

Anders als die Modellumgebung müssen die Modellobjekte nicht als reale Objekte vorhanden sein, weshalb die Modellgüte auch keine Auswirkungen auf die Lokalisations- oder Projektionsgenauigkeit hat. Es können somit beliebige Objekte modelliert werden, um diese später in der realen Umgebung zu visualisieren.  Für die Visualisierung ist jedoch eine ausreichende Auflösung der Modellstrukturen und -texturen anzustreben. \abb{fig.modobj} zeigt beispielhafte Modellobjekte und ihre Integration in die Modellumgebung. Die zu visualisierenden Modellobjekte werden anhand der virtuellen Umgebungsdaten orientiert und positioniert und können damit über die im Rahmen der Lokalisation durchgeführte Bestimmung der Systempose in die reale Umgebung projiziert werden.

\prever{
%\red[Zweite Steckdose in Bild einfügen\\]
}

\begin{figure}[ht]
	\begin{center}
	\includegraphics[scale=1.0]{model_obj_pc_02}
	\caption{Integration virtueller Modellobjekte in die Modellumgebung}
	\label{fig.modobj}
	\end{center}
	%\vspace*{-8mm}
\end{figure}

%\begin{figure}[ht]
%	\begin{center}
%	\subfigure[Modellobjekt]{
%		\begingroup\fboxsep=5pt\fboxrule=0pt
%		\fbox{%
%			\includegraphics[scale=0.05]{steckdose}
%		}
%		\endgroup
%	}
%	\hspace{3mm}
%%	\subfigure[Modellobjekt elektrische Leitung]{
%%			\includegraphics[scale=1.0]{spacer}
%%	}
%%	\hspace{3mm}
%	\subfigure[In die Modellumgebung integriertes Modellobjekt]{
%		\includegraphics[scale=0.2]{model_obj_pointcloud}
%	}
%	\caption{Integration eines virtuellen Modellobjektes in die Modellumgebung}
%	\label{fig.modobj}
%	\end{center}
%	%\vspace*{-8mm}
%\end{figure}

%\red[Modellobjekte ruhig in CAD Umgebung einfügen!\\Vielleicht das Bild von später nehmen?\\]

\section{Datenstruktur}
Um die Szene, bestehend aus der Modellumgebung und allen integrierten Modellobjekten, später wieder in den virtuellen Planungsablauf integrieren zu können, wird eine Datenstruktur definiert, welche die Konfiguration der Modellszene repräsentiert. Alle Modellelemente werden dazu mit ihrer aktuellen Pose, Textur und geometrischen Repräsentation erfasst, um das Sichern und Wiederherstellen von Szenarien zu ermöglichen. Ein Beispiel, welches den Aufbau der Datenstruktur beschreibt, findet sich in \anhang{app.datastructure}\\