\chapter{Lokalisation}

\section{Lokalisationsverfahren}

\section{Ermittlung der Initialpose}
Globale Lokalisation\\
Für die globale Lokalisation wurde in dieser Arbeit ein Ansatz gewählt, welcher auf dem Monte Carlo Algorithmus basiert. Dieser Ansatz entspricht wie bereits in Kapitel \ref{chap:mcl} beschrieben einem Partikelfilter. In der vorhandenen Umgebungskarte werden im Rahmen zuvor festlegbarer Parameter Partikel basierend auf einer Zufallsverteilung generiert. Jeder Partikel entspricht dabei einer möglichen Pose des Kamera-Projektor-Systems. Jeder Partikel verfügt demnach über sechs Freiheitsgrade. Die Bestimmung der Pose des Kamera-Projektor-Systems erfolgt durch Auswertung der Partikel basierend auf einer Wahrscheinlichkeitsverteilung. Die Wahrscheinlichkeiten können dabei auf Basis zwei verschiedener Modelle berechnet werden, dem Endpoint-Modell und dem Raycasting-Modell.\\
Beim Endpoint-Modell, welches von X in \cite{Endpoint} beschrieben wurde, erfolgt eine Bestimmung der Wahrscheinlichkeit der Pose basierend auf einem Distanz-Modell, welches für die verwendete Karte  statisch ist und damit bereits vor der Lokalisation berechnet werden kann. Dadurch ergibt sich eine deutliche Verringerung des Rechenaufwands bei jedem Lokalisierungsschritt. Betrachtet werden beim Endpoint-Modell lediglich die äußersten Punkte der Karte, das bedeutet, das Hindernisse innerhalb der Karte nicht berücksichtigt werden. (?)\\
Beim Raycasting Modell, welches ebenfalls von X in \cite{Raycasting} beschrieben wird, wird für jede Pose eine Betrachtung der Strahlen durchgeführt, welche an dieser Stelle theoretisch in Richtung der Messpunkte ausgesendet werden würden. Entlang dieser Strahlen erfolgt dann ein Abgleich mit den Daten der Karte, so dass die Punkte ermittelt werden können, an denen die Strahlen auf ein Hindernis in der Karte treffen. Die so ermittelten Daten können dann mit den Sensordaten abgeglichen werden, um daraus die Wahrscheinlichkeit der Pose des aktuellen Partikels zu bestimmen.\\
Markerbasierte Lokalisation\\

\section{Verfolgung der aktuellen Pose}
Partikelbasiertes Tracking\\
Featurebasiertes Tracking\\
Beschleunigungsdatenbasiertes Tracking\\
Kombination der Odometriedaten durch Extended Kalman Filter\\