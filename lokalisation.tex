\chapter{Lokalisation}

\section{Lokalisationsverfahren}

\section{Ermittlung der Initialpose}
Globale Lokalisation\\
Für die globale Lokalisation wurde in dieser Arbeit ein Ansatz gewählt, welcher auf dem Monte Carlo Algorithmus basiert. Dieser Ansatz entspricht wie bereits in Kapitel \ref{chap:mcl} beschrieben einem Partikelfilter. In der vorhandenen Umgebungskarte werden im Rahmen zuvor festlegbarer Parameter Partikel basierend auf einer Zufallsverteilung generiert. Jeder Partikel entspricht dabei einer möglichen Pose des Kamera-Projektor-Systems und verfügt demnach über sechs Freiheitsgrade. Die Bestimmung der Pose des Kamera-Projektor-Systems erfolgt durch Auswertung der Partikel basierend auf einer Wahrscheinlichkeitsverteilung. Die Wahrscheinlichkeiten können dabei auf Basis zwei verschiedener Modelle berechnet werden, dem Endpoint-Modell und dem Raycasting-Modell.\\
Beim Endpoint-Modell, welches von X in \cite{Endpoint} beschrieben wurde, erfolgt eine Bestimmung der Wahrscheinlichkeit der Pose basierend auf einem Distanz-Modell, welches für die verwendete Karte statisch ist und damit bereits vor der Lokalisation berechnet werden kann. Dabei wird basierend auf der für die Lokalisation verwendeten Karte eine Lookup-Table erstellt welche die räumlichen Dimensionen der Ausgangskarte abbildet. Jedem Voxel wird dabei ein Distanzwert zugewiesen, basierend auf der Entfernung zum nächsten Hindernis. Als Hindernisse gelten dabei neben dem Raum auch alle statischen Objekte, welche in der Karte vorhanden sind.\\

\begin{figure}[!ht]
	\begin{center}
		\includegraphics[scale=1.0]{spacer}
		\caption{Distance Map}
		\label{fig.dist_map}
	\end{center}
	%\vspace*{-8mm}
\end{figure}

Durch die vorhergehende Berechnung dieser Karte ergibt sich eine deutliche Verringerung des Rechenaufwands bei jedem Lokalisierungsschritt. Anzumerken bleibt, dass die Distanz-Karte während der gesamten Programmlaufdauer im Speicher behalten wird und bezüglich des Speicherbedarfs die Ausgangskarte deutlich übersteigen kann. Die Lokalisation erfolgt beim beim Endpoint-Modell dann durch Betrachtung der Endpunkte des Tiefenbildes, also der Punkte, welche durch die aufgenommene Punktewolke in Abhängigkeit des betrachteten Partikels innerhalb der Karte abgebildet werden. An diesen Punkten werden die in der zuvor berechneten Distanz-Karte gespeicherten Werte ausgelesen und darauf basierend die Wahrscheinlichkeit berechnet, dass die Punktewolke von der Stelle des betrachteten Partikels aus aufgenommen wurde. Die Berechnung erfolgt mittels der Log-Likelihood-Funktion, welche eine statistische Aussage über die Wahrscheinlichkeit der Messung bestimmt.\\
Beim Raycasting Modell, welches ebenfalls von X in \cite{Raycasting} beschrieben wird, wird für jede Pose eine Betrachtung der Strahlen durchgeführt, welche an dieser Stelle theoretisch in Richtung der Messpunkte ausgesendet werden würden. Entlang dieser Strahlen findet ein Abgleich mit den Daten der Karte statt. Erreicht der Strahl einen besetzen Bereich in der Karte endet die Betrachtung des Strahls an diesem Punkt.\\

\begin{figure}[!ht]
	\begin{center}
		\includegraphics[scale=1.0]{spacer}
		\caption{Raycasting-Modell}
		\label{fig.raycast}
	\end{center}
	%\vspace*{-8mm}
\end{figure}

Die Entfernung zwischen dem so ermittelten Punkt und dem Betrachteten Partikel wird berechnet und anschließend mit der Entfernung des zugehörigen Sensorwertes verglichen. Die Differenz der beiden Distanzen wird anschließend ermittelt um die Wahrscheinlichkeit zu bestimmen, dass der Sensorwert an dieser Stelle aufgenommen wurde. Die Berechnung der Wahrscheinlichkeit erfolgt dabei ebenfalls mittels der Log-Likelihood-Funktion.\\
\red[LOG-Likelihood Funktion aufführen und beschreiben]

Markerbasierte Lokalisation\\

\section{Verfolgung der aktuellen Pose}
Partikelbasiertes Tracking\\
Featurebasiertes Tracking\\
Beschleunigungsdatenbasiertes Tracking\\
Kombination der Odometriedaten durch Extended Kalman Filter \red[Buch Hertzberg]\\